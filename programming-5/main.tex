\documentclass[11pt]{article}

    \usepackage[breakable]{tcolorbox}
    \usepackage{parskip} % Stop auto-indenting (to mimic markdown behaviour)
    

    % Basic figure setup, for now with no caption control since it's done
    % automatically by Pandoc (which extracts ![](path) syntax from Markdown).
    \usepackage{graphicx}
    % Maintain compatibility with old templates. Remove in nbconvert 6.0
    \let\Oldincludegraphics\includegraphics
    % Ensure that by default, figures have no caption (until we provide a
    % proper Figure object with a Caption API and a way to capture that
    % in the conversion process - todo).
    \usepackage{caption}
    \DeclareCaptionFormat{nocaption}{}
    \captionsetup{format=nocaption,aboveskip=0pt,belowskip=0pt}

    \usepackage{float}
    \floatplacement{figure}{H} % forces figures to be placed at the correct location
    \usepackage{xcolor} % Allow colors to be defined
    \usepackage{enumerate} % Needed for markdown enumerations to work
    \usepackage{geometry} % Used to adjust the document margins
    \usepackage{amsmath} % Equations
    \usepackage{amssymb} % Equations
    \usepackage{textcomp} % defines textquotesingle
    % Hack from http://tex.stackexchange.com/a/47451/13684:
    \AtBeginDocument{%
        \def\PYZsq{\textquotesingle}% Upright quotes in Pygmentized code
    }
    \usepackage{upquote} % Upright quotes for verbatim code
    \usepackage{eurosym} % defines \euro

    \usepackage{iftex}
    \ifPDFTeX
        \usepackage[T1]{fontenc}
        \IfFileExists{alphabeta.sty}{
              \usepackage{alphabeta}
          }{
              \usepackage[mathletters]{ucs}
              \usepackage[utf8x]{inputenc}
          }
    \else
        \usepackage{fontspec}
        \usepackage{unicode-math}
    \fi

    \usepackage{fancyvrb} % verbatim replacement that allows latex
    \usepackage{grffile} % extends the file name processing of package graphics 
                         % to support a larger range
    \makeatletter % fix for old versions of grffile with XeLaTeX
    \@ifpackagelater{grffile}{2019/11/01}
    {
      % Do nothing on new versions
    }
    {
      \def\Gread@@xetex#1{%
        \IfFileExists{"\Gin@base".bb}%
        {\Gread@eps{\Gin@base.bb}}%
        {\Gread@@xetex@aux#1}%
      }
    }
    \makeatother
    \usepackage[Export]{adjustbox} % Used to constrain images to a maximum size
    \adjustboxset{max size={0.9\linewidth}{0.9\paperheight}}

    % The hyperref package gives us a pdf with properly built
    % internal navigation ('pdf bookmarks' for the table of contents,
    % internal cross-reference links, web links for URLs, etc.)
    \usepackage{hyperref}
    % The default LaTeX title has an obnoxious amount of whitespace. By default,
    % titling removes some of it. It also provides customization options.
    \usepackage{titling}
    \usepackage{longtable} % longtable support required by pandoc >1.10
    \usepackage{booktabs}  % table support for pandoc > 1.12.2
    \usepackage{array}     % table support for pandoc >= 2.11.3
    \usepackage{calc}      % table minipage width calculation for pandoc >= 2.11.1
    \usepackage[inline]{enumitem} % IRkernel/repr support (it uses the enumerate* environment)
    \usepackage[normalem]{ulem} % ulem is needed to support strikethroughs (\sout)
                                % normalem makes italics be italics, not underlines
    \usepackage{mathrsfs}
    

    
    % Colors for the hyperref package
    \definecolor{urlcolor}{rgb}{0,.145,.698}
    \definecolor{linkcolor}{rgb}{.71,0.21,0.01}
    \definecolor{citecolor}{rgb}{.12,.54,.11}

    % ANSI colors
    \definecolor{ansi-black}{HTML}{3E424D}
    \definecolor{ansi-black-intense}{HTML}{282C36}
    \definecolor{ansi-red}{HTML}{E75C58}
    \definecolor{ansi-red-intense}{HTML}{B22B31}
    \definecolor{ansi-green}{HTML}{00A250}
    \definecolor{ansi-green-intense}{HTML}{007427}
    \definecolor{ansi-yellow}{HTML}{DDB62B}
    \definecolor{ansi-yellow-intense}{HTML}{B27D12}
    \definecolor{ansi-blue}{HTML}{208FFB}
    \definecolor{ansi-blue-intense}{HTML}{0065CA}
    \definecolor{ansi-magenta}{HTML}{D160C4}
    \definecolor{ansi-magenta-intense}{HTML}{A03196}
    \definecolor{ansi-cyan}{HTML}{60C6C8}
    \definecolor{ansi-cyan-intense}{HTML}{258F8F}
    \definecolor{ansi-white}{HTML}{C5C1B4}
    \definecolor{ansi-white-intense}{HTML}{A1A6B2}
    \definecolor{ansi-default-inverse-fg}{HTML}{FFFFFF}
    \definecolor{ansi-default-inverse-bg}{HTML}{000000}

    % common color for the border for error outputs.
    \definecolor{outerrorbackground}{HTML}{FFDFDF}

    % commands and environments needed by pandoc snippets
    % extracted from the output of `pandoc -s`
    \providecommand{\tightlist}{%
      \setlength{\itemsep}{0pt}\setlength{\parskip}{0pt}}
    \DefineVerbatimEnvironment{Highlighting}{Verbatim}{commandchars=\\\{\}}
    % Add ',fontsize=\small' for more characters per line
    \newenvironment{Shaded}{}{}
    \newcommand{\KeywordTok}[1]{\textcolor[rgb]{0.00,0.44,0.13}{\textbf{{#1}}}}
    \newcommand{\DataTypeTok}[1]{\textcolor[rgb]{0.56,0.13,0.00}{{#1}}}
    \newcommand{\DecValTok}[1]{\textcolor[rgb]{0.25,0.63,0.44}{{#1}}}
    \newcommand{\BaseNTok}[1]{\textcolor[rgb]{0.25,0.63,0.44}{{#1}}}
    \newcommand{\FloatTok}[1]{\textcolor[rgb]{0.25,0.63,0.44}{{#1}}}
    \newcommand{\CharTok}[1]{\textcolor[rgb]{0.25,0.44,0.63}{{#1}}}
    \newcommand{\StringTok}[1]{\textcolor[rgb]{0.25,0.44,0.63}{{#1}}}
    \newcommand{\CommentTok}[1]{\textcolor[rgb]{0.38,0.63,0.69}{\textit{{#1}}}}
    \newcommand{\OtherTok}[1]{\textcolor[rgb]{0.00,0.44,0.13}{{#1}}}
    \newcommand{\AlertTok}[1]{\textcolor[rgb]{1.00,0.00,0.00}{\textbf{{#1}}}}
    \newcommand{\FunctionTok}[1]{\textcolor[rgb]{0.02,0.16,0.49}{{#1}}}
    \newcommand{\RegionMarkerTok}[1]{{#1}}
    \newcommand{\ErrorTok}[1]{\textcolor[rgb]{1.00,0.00,0.00}{\textbf{{#1}}}}
    \newcommand{\NormalTok}[1]{{#1}}
    
    % Additional commands for more recent versions of Pandoc
    \newcommand{\ConstantTok}[1]{\textcolor[rgb]{0.53,0.00,0.00}{{#1}}}
    \newcommand{\SpecialCharTok}[1]{\textcolor[rgb]{0.25,0.44,0.63}{{#1}}}
    \newcommand{\VerbatimStringTok}[1]{\textcolor[rgb]{0.25,0.44,0.63}{{#1}}}
    \newcommand{\SpecialStringTok}[1]{\textcolor[rgb]{0.73,0.40,0.53}{{#1}}}
    \newcommand{\ImportTok}[1]{{#1}}
    \newcommand{\DocumentationTok}[1]{\textcolor[rgb]{0.73,0.13,0.13}{\textit{{#1}}}}
    \newcommand{\AnnotationTok}[1]{\textcolor[rgb]{0.38,0.63,0.69}{\textbf{\textit{{#1}}}}}
    \newcommand{\CommentVarTok}[1]{\textcolor[rgb]{0.38,0.63,0.69}{\textbf{\textit{{#1}}}}}
    \newcommand{\VariableTok}[1]{\textcolor[rgb]{0.10,0.09,0.49}{{#1}}}
    \newcommand{\ControlFlowTok}[1]{\textcolor[rgb]{0.00,0.44,0.13}{\textbf{{#1}}}}
    \newcommand{\OperatorTok}[1]{\textcolor[rgb]{0.40,0.40,0.40}{{#1}}}
    \newcommand{\BuiltInTok}[1]{{#1}}
    \newcommand{\ExtensionTok}[1]{{#1}}
    \newcommand{\PreprocessorTok}[1]{\textcolor[rgb]{0.74,0.48,0.00}{{#1}}}
    \newcommand{\AttributeTok}[1]{\textcolor[rgb]{0.49,0.56,0.16}{{#1}}}
    \newcommand{\InformationTok}[1]{\textcolor[rgb]{0.38,0.63,0.69}{\textbf{\textit{{#1}}}}}
    \newcommand{\WarningTok}[1]{\textcolor[rgb]{0.38,0.63,0.69}{\textbf{\textit{{#1}}}}}
    
    
    % Define a nice break command that doesn't care if a line doesn't already
    % exist.
    \def\br{\hspace*{\fill} \\* }
    % Math Jax compatibility definitions
    \def\gt{>}
    \def\lt{<}
    \let\Oldtex\TeX
    \let\Oldlatex\LaTeX
    \renewcommand{\TeX}{\textrm{\Oldtex}}
    \renewcommand{\LaTeX}{\textrm{\Oldlatex}}
    % Document parameters
    % Document title
    \title{AI2619 - Programming Assignment 5}
    \author{Yikun Ji}
    
    
    
    
% Pygments definitions
\makeatletter
\def\PY@reset{\let\PY@it=\relax \let\PY@bf=\relax%
    \let\PY@ul=\relax \let\PY@tc=\relax%
    \let\PY@bc=\relax \let\PY@ff=\relax}
\def\PY@tok#1{\csname PY@tok@#1\endcsname}
\def\PY@toks#1+{\ifx\relax#1\empty\else%
    \PY@tok{#1}\expandafter\PY@toks\fi}
\def\PY@do#1{\PY@bc{\PY@tc{\PY@ul{%
    \PY@it{\PY@bf{\PY@ff{#1}}}}}}}
\def\PY#1#2{\PY@reset\PY@toks#1+\relax+\PY@do{#2}}

\@namedef{PY@tok@w}{\def\PY@tc##1{\textcolor[rgb]{0.73,0.73,0.73}{##1}}}
\@namedef{PY@tok@c}{\let\PY@it=\textit\def\PY@tc##1{\textcolor[rgb]{0.24,0.48,0.48}{##1}}}
\@namedef{PY@tok@cp}{\def\PY@tc##1{\textcolor[rgb]{0.61,0.40,0.00}{##1}}}
\@namedef{PY@tok@k}{\let\PY@bf=\textbf\def\PY@tc##1{\textcolor[rgb]{0.00,0.50,0.00}{##1}}}
\@namedef{PY@tok@kp}{\def\PY@tc##1{\textcolor[rgb]{0.00,0.50,0.00}{##1}}}
\@namedef{PY@tok@kt}{\def\PY@tc##1{\textcolor[rgb]{0.69,0.00,0.25}{##1}}}
\@namedef{PY@tok@o}{\def\PY@tc##1{\textcolor[rgb]{0.40,0.40,0.40}{##1}}}
\@namedef{PY@tok@ow}{\let\PY@bf=\textbf\def\PY@tc##1{\textcolor[rgb]{0.67,0.13,1.00}{##1}}}
\@namedef{PY@tok@nb}{\def\PY@tc##1{\textcolor[rgb]{0.00,0.50,0.00}{##1}}}
\@namedef{PY@tok@nf}{\def\PY@tc##1{\textcolor[rgb]{0.00,0.00,1.00}{##1}}}
\@namedef{PY@tok@nc}{\let\PY@bf=\textbf\def\PY@tc##1{\textcolor[rgb]{0.00,0.00,1.00}{##1}}}
\@namedef{PY@tok@nn}{\let\PY@bf=\textbf\def\PY@tc##1{\textcolor[rgb]{0.00,0.00,1.00}{##1}}}
\@namedef{PY@tok@ne}{\let\PY@bf=\textbf\def\PY@tc##1{\textcolor[rgb]{0.80,0.25,0.22}{##1}}}
\@namedef{PY@tok@nv}{\def\PY@tc##1{\textcolor[rgb]{0.10,0.09,0.49}{##1}}}
\@namedef{PY@tok@no}{\def\PY@tc##1{\textcolor[rgb]{0.53,0.00,0.00}{##1}}}
\@namedef{PY@tok@nl}{\def\PY@tc##1{\textcolor[rgb]{0.46,0.46,0.00}{##1}}}
\@namedef{PY@tok@ni}{\let\PY@bf=\textbf\def\PY@tc##1{\textcolor[rgb]{0.44,0.44,0.44}{##1}}}
\@namedef{PY@tok@na}{\def\PY@tc##1{\textcolor[rgb]{0.41,0.47,0.13}{##1}}}
\@namedef{PY@tok@nt}{\let\PY@bf=\textbf\def\PY@tc##1{\textcolor[rgb]{0.00,0.50,0.00}{##1}}}
\@namedef{PY@tok@nd}{\def\PY@tc##1{\textcolor[rgb]{0.67,0.13,1.00}{##1}}}
\@namedef{PY@tok@s}{\def\PY@tc##1{\textcolor[rgb]{0.73,0.13,0.13}{##1}}}
\@namedef{PY@tok@sd}{\let\PY@it=\textit\def\PY@tc##1{\textcolor[rgb]{0.73,0.13,0.13}{##1}}}
\@namedef{PY@tok@si}{\let\PY@bf=\textbf\def\PY@tc##1{\textcolor[rgb]{0.64,0.35,0.47}{##1}}}
\@namedef{PY@tok@se}{\let\PY@bf=\textbf\def\PY@tc##1{\textcolor[rgb]{0.67,0.36,0.12}{##1}}}
\@namedef{PY@tok@sr}{\def\PY@tc##1{\textcolor[rgb]{0.64,0.35,0.47}{##1}}}
\@namedef{PY@tok@ss}{\def\PY@tc##1{\textcolor[rgb]{0.10,0.09,0.49}{##1}}}
\@namedef{PY@tok@sx}{\def\PY@tc##1{\textcolor[rgb]{0.00,0.50,0.00}{##1}}}
\@namedef{PY@tok@m}{\def\PY@tc##1{\textcolor[rgb]{0.40,0.40,0.40}{##1}}}
\@namedef{PY@tok@gh}{\let\PY@bf=\textbf\def\PY@tc##1{\textcolor[rgb]{0.00,0.00,0.50}{##1}}}
\@namedef{PY@tok@gu}{\let\PY@bf=\textbf\def\PY@tc##1{\textcolor[rgb]{0.50,0.00,0.50}{##1}}}
\@namedef{PY@tok@gd}{\def\PY@tc##1{\textcolor[rgb]{0.63,0.00,0.00}{##1}}}
\@namedef{PY@tok@gi}{\def\PY@tc##1{\textcolor[rgb]{0.00,0.52,0.00}{##1}}}
\@namedef{PY@tok@gr}{\def\PY@tc##1{\textcolor[rgb]{0.89,0.00,0.00}{##1}}}
\@namedef{PY@tok@ge}{\let\PY@it=\textit}
\@namedef{PY@tok@gs}{\let\PY@bf=\textbf}
\@namedef{PY@tok@gp}{\let\PY@bf=\textbf\def\PY@tc##1{\textcolor[rgb]{0.00,0.00,0.50}{##1}}}
\@namedef{PY@tok@go}{\def\PY@tc##1{\textcolor[rgb]{0.44,0.44,0.44}{##1}}}
\@namedef{PY@tok@gt}{\def\PY@tc##1{\textcolor[rgb]{0.00,0.27,0.87}{##1}}}
\@namedef{PY@tok@err}{\def\PY@bc##1{{\setlength{\fboxsep}{\string -\fboxrule}\fcolorbox[rgb]{1.00,0.00,0.00}{1,1,1}{\strut ##1}}}}
\@namedef{PY@tok@kc}{\let\PY@bf=\textbf\def\PY@tc##1{\textcolor[rgb]{0.00,0.50,0.00}{##1}}}
\@namedef{PY@tok@kd}{\let\PY@bf=\textbf\def\PY@tc##1{\textcolor[rgb]{0.00,0.50,0.00}{##1}}}
\@namedef{PY@tok@kn}{\let\PY@bf=\textbf\def\PY@tc##1{\textcolor[rgb]{0.00,0.50,0.00}{##1}}}
\@namedef{PY@tok@kr}{\let\PY@bf=\textbf\def\PY@tc##1{\textcolor[rgb]{0.00,0.50,0.00}{##1}}}
\@namedef{PY@tok@bp}{\def\PY@tc##1{\textcolor[rgb]{0.00,0.50,0.00}{##1}}}
\@namedef{PY@tok@fm}{\def\PY@tc##1{\textcolor[rgb]{0.00,0.00,1.00}{##1}}}
\@namedef{PY@tok@vc}{\def\PY@tc##1{\textcolor[rgb]{0.10,0.09,0.49}{##1}}}
\@namedef{PY@tok@vg}{\def\PY@tc##1{\textcolor[rgb]{0.10,0.09,0.49}{##1}}}
\@namedef{PY@tok@vi}{\def\PY@tc##1{\textcolor[rgb]{0.10,0.09,0.49}{##1}}}
\@namedef{PY@tok@vm}{\def\PY@tc##1{\textcolor[rgb]{0.10,0.09,0.49}{##1}}}
\@namedef{PY@tok@sa}{\def\PY@tc##1{\textcolor[rgb]{0.73,0.13,0.13}{##1}}}
\@namedef{PY@tok@sb}{\def\PY@tc##1{\textcolor[rgb]{0.73,0.13,0.13}{##1}}}
\@namedef{PY@tok@sc}{\def\PY@tc##1{\textcolor[rgb]{0.73,0.13,0.13}{##1}}}
\@namedef{PY@tok@dl}{\def\PY@tc##1{\textcolor[rgb]{0.73,0.13,0.13}{##1}}}
\@namedef{PY@tok@s2}{\def\PY@tc##1{\textcolor[rgb]{0.73,0.13,0.13}{##1}}}
\@namedef{PY@tok@sh}{\def\PY@tc##1{\textcolor[rgb]{0.73,0.13,0.13}{##1}}}
\@namedef{PY@tok@s1}{\def\PY@tc##1{\textcolor[rgb]{0.73,0.13,0.13}{##1}}}
\@namedef{PY@tok@mb}{\def\PY@tc##1{\textcolor[rgb]{0.40,0.40,0.40}{##1}}}
\@namedef{PY@tok@mf}{\def\PY@tc##1{\textcolor[rgb]{0.40,0.40,0.40}{##1}}}
\@namedef{PY@tok@mh}{\def\PY@tc##1{\textcolor[rgb]{0.40,0.40,0.40}{##1}}}
\@namedef{PY@tok@mi}{\def\PY@tc##1{\textcolor[rgb]{0.40,0.40,0.40}{##1}}}
\@namedef{PY@tok@il}{\def\PY@tc##1{\textcolor[rgb]{0.40,0.40,0.40}{##1}}}
\@namedef{PY@tok@mo}{\def\PY@tc##1{\textcolor[rgb]{0.40,0.40,0.40}{##1}}}
\@namedef{PY@tok@ch}{\let\PY@it=\textit\def\PY@tc##1{\textcolor[rgb]{0.24,0.48,0.48}{##1}}}
\@namedef{PY@tok@cm}{\let\PY@it=\textit\def\PY@tc##1{\textcolor[rgb]{0.24,0.48,0.48}{##1}}}
\@namedef{PY@tok@cpf}{\let\PY@it=\textit\def\PY@tc##1{\textcolor[rgb]{0.24,0.48,0.48}{##1}}}
\@namedef{PY@tok@c1}{\let\PY@it=\textit\def\PY@tc##1{\textcolor[rgb]{0.24,0.48,0.48}{##1}}}
\@namedef{PY@tok@cs}{\let\PY@it=\textit\def\PY@tc##1{\textcolor[rgb]{0.24,0.48,0.48}{##1}}}

\def\PYZbs{\char`\\}
\def\PYZus{\char`\_}
\def\PYZob{\char`\{}
\def\PYZcb{\char`\}}
\def\PYZca{\char`\^}
\def\PYZam{\char`\&}
\def\PYZlt{\char`\<}
\def\PYZgt{\char`\>}
\def\PYZsh{\char`\#}
\def\PYZpc{\char`\%}
\def\PYZdl{\char`\$}
\def\PYZhy{\char`\-}
\def\PYZsq{\char`\'}
\def\PYZdq{\char`\"}
\def\PYZti{\char`\~}
% for compatibility with earlier versions
\def\PYZat{@}
\def\PYZlb{[}
\def\PYZrb{]}
\makeatother


    % For linebreaks inside Verbatim environment from package fancyvrb. 
    \makeatletter
        \newbox\Wrappedcontinuationbox 
        \newbox\Wrappedvisiblespacebox 
        \newcommand*\Wrappedvisiblespace {\textcolor{red}{\textvisiblespace}} 
        \newcommand*\Wrappedcontinuationsymbol {\textcolor{red}{\llap{\tiny$\m@th\hookrightarrow$}}} 
        \newcommand*\Wrappedcontinuationindent {3ex } 
        \newcommand*\Wrappedafterbreak {\kern\Wrappedcontinuationindent\copy\Wrappedcontinuationbox} 
        % Take advantage of the already applied Pygments mark-up to insert 
        % potential linebreaks for TeX processing. 
        %        {, <, #, %, $, ' and ": go to next line. 
        %        _, }, ^, &, >, - and ~: stay at end of broken line. 
        % Use of \textquotesingle for straight quote. 
        \newcommand*\Wrappedbreaksatspecials {% 
            \def\PYGZus{\discretionary{\char`\_}{\Wrappedafterbreak}{\char`\_}}% 
            \def\PYGZob{\discretionary{}{\Wrappedafterbreak\char`\{}{\char`\{}}% 
            \def\PYGZcb{\discretionary{\char`\}}{\Wrappedafterbreak}{\char`\}}}% 
            \def\PYGZca{\discretionary{\char`\^}{\Wrappedafterbreak}{\char`\^}}% 
            \def\PYGZam{\discretionary{\char`\&}{\Wrappedafterbreak}{\char`\&}}% 
            \def\PYGZlt{\discretionary{}{\Wrappedafterbreak\char`\<}{\char`\<}}% 
            \def\PYGZgt{\discretionary{\char`\>}{\Wrappedafterbreak}{\char`\>}}% 
            \def\PYGZsh{\discretionary{}{\Wrappedafterbreak\char`\#}{\char`\#}}% 
            \def\PYGZpc{\discretionary{}{\Wrappedafterbreak\char`\%}{\char`\%}}% 
            \def\PYGZdl{\discretionary{}{\Wrappedafterbreak\char`\$}{\char`\$}}% 
            \def\PYGZhy{\discretionary{\char`\-}{\Wrappedafterbreak}{\char`\-}}% 
            \def\PYGZsq{\discretionary{}{\Wrappedafterbreak\textquotesingle}{\textquotesingle}}% 
            \def\PYGZdq{\discretionary{}{\Wrappedafterbreak\char`\"}{\char`\"}}% 
            \def\PYGZti{\discretionary{\char`\~}{\Wrappedafterbreak}{\char`\~}}% 
        } 
        % Some characters . , ; ? ! / are not pygmentized. 
        % This macro makes them "active" and they will insert potential linebreaks 
        \newcommand*\Wrappedbreaksatpunct {% 
            \lccode`\~`\.\lowercase{\def~}{\discretionary{\hbox{\char`\.}}{\Wrappedafterbreak}{\hbox{\char`\.}}}% 
            \lccode`\~`\,\lowercase{\def~}{\discretionary{\hbox{\char`\,}}{\Wrappedafterbreak}{\hbox{\char`\,}}}% 
            \lccode`\~`\;\lowercase{\def~}{\discretionary{\hbox{\char`\;}}{\Wrappedafterbreak}{\hbox{\char`\;}}}% 
            \lccode`\~`\:\lowercase{\def~}{\discretionary{\hbox{\char`\:}}{\Wrappedafterbreak}{\hbox{\char`\:}}}% 
            \lccode`\~`\?\lowercase{\def~}{\discretionary{\hbox{\char`\?}}{\Wrappedafterbreak}{\hbox{\char`\?}}}% 
            \lccode`\~`\!\lowercase{\def~}{\discretionary{\hbox{\char`\!}}{\Wrappedafterbreak}{\hbox{\char`\!}}}% 
            \lccode`\~`\/\lowercase{\def~}{\discretionary{\hbox{\char`\/}}{\Wrappedafterbreak}{\hbox{\char`\/}}}% 
            \catcode`\.\active
            \catcode`\,\active 
            \catcode`\;\active
            \catcode`\:\active
            \catcode`\?\active
            \catcode`\!\active
            \catcode`\/\active 
            \lccode`\~`\~ 	
        }
    \makeatother

    \let\OriginalVerbatim=\Verbatim
    \makeatletter
    \renewcommand{\Verbatim}[1][1]{%
        %\parskip\z@skip
        \sbox\Wrappedcontinuationbox {\Wrappedcontinuationsymbol}%
        \sbox\Wrappedvisiblespacebox {\FV@SetupFont\Wrappedvisiblespace}%
        \def\FancyVerbFormatLine ##1{\hsize\linewidth
            \vtop{\raggedright\hyphenpenalty\z@\exhyphenpenalty\z@
                \doublehyphendemerits\z@\finalhyphendemerits\z@
                \strut ##1\strut}%
        }%
        % If the linebreak is at a space, the latter will be displayed as visible
        % space at end of first line, and a continuation symbol starts next line.
        % Stretch/shrink are however usually zero for typewriter font.
        \def\FV@Space {%
            \nobreak\hskip\z@ plus\fontdimen3\font minus\fontdimen4\font
            \discretionary{\copy\Wrappedvisiblespacebox}{\Wrappedafterbreak}
            {\kern\fontdimen2\font}%
        }%
        
        % Allow breaks at special characters using \PYG... macros.
        \Wrappedbreaksatspecials
        % Breaks at punctuation characters . , ; ? ! and / need catcode=\active 	
        \OriginalVerbatim[#1,codes*=\Wrappedbreaksatpunct]%
    }
    \makeatother

    % Exact colors from NB
    \definecolor{incolor}{HTML}{303F9F}
    \definecolor{outcolor}{HTML}{D84315}
    \definecolor{cellborder}{HTML}{CFCFCF}
    \definecolor{cellbackground}{HTML}{F7F7F7}
    
    % prompt
    \makeatletter
    \newcommand{\boxspacing}{\kern\kvtcb@left@rule\kern\kvtcb@boxsep}
    \makeatother
    \newcommand{\prompt}[4]{
        {\ttfamily\llap{{\color{#2}[#3]:\hspace{3pt}#4}}\vspace{-\baselineskip}}
    }
    

    
    % Prevent overflowing lines due to hard-to-break entities
    \sloppy 
    % Setup hyperref package
    \hypersetup{
      breaklinks=true,  % so long urls are correctly broken across lines
      colorlinks=true,
      urlcolor=urlcolor,
      linkcolor=linkcolor,
      citecolor=citecolor,
      }
    % Slightly bigger margins than the latex defaults
    
    \geometry{verbose,tmargin=1in,bmargin=1in,lmargin=1in,rmargin=1in}
    
    

\begin{document}
    
    \maketitle
    
    

    
    This homework is mainly about image processing, where we will enhance
and filter a specific image.

    \hypertarget{preparations}{%
\section{Preparations}\label{preparations}}

The target image will be converted to a numpy matrix using
\texttt{imageio}.

    \begin{tcolorbox}[breakable, size=fbox, boxrule=1pt, pad at break*=1mm,colback=cellbackground, colframe=cellborder]
\prompt{In}{incolor}{51}{\boxspacing}
\begin{Verbatim}[commandchars=\\\{\}]
\PY{k+kn}{import} \PY{n+nn}{imageio}
\PY{k+kn}{from} \PY{n+nn}{skimage} \PY{k+kn}{import} \PY{n}{exposure}
\PY{k+kn}{import} \PY{n+nn}{matplotlib}\PY{n+nn}{.}\PY{n+nn}{pyplot} \PY{k}{as} \PY{n+nn}{plt}
\PY{k+kn}{import} \PY{n+nn}{numpy} \PY{k}{as} \PY{n+nn}{np}
\PY{k+kn}{import} \PY{n+nn}{cv2}
\PY{o}{\PYZpc{}}\PY{k}{matplotlib} inline
\PY{o}{\PYZpc{}}\PY{k}{config} InlineBackend.rc = \PYZob{}\PYZsq{}figure.dpi\PYZsq{}: 320\PYZcb{}
\PY{o}{\PYZpc{}}\PY{k}{config} InlineBackend.figure\PYZus{}format = \PYZsq{}png\PYZsq{}
\PY{n}{plt}\PY{o}{.}\PY{n}{rc}\PY{p}{(}\PY{l+s+s1}{\PYZsq{}}\PY{l+s+s1}{text}\PY{l+s+s1}{\PYZsq{}}\PY{p}{,} \PY{n}{usetex}\PY{o}{=}\PY{k+kc}{True}\PY{p}{)}
\PY{n}{plt}\PY{o}{.}\PY{n}{rc}\PY{p}{(}\PY{l+s+s1}{\PYZsq{}}\PY{l+s+s1}{font}\PY{l+s+s1}{\PYZsq{}}\PY{p}{,} \PY{n}{family}\PY{o}{=}\PY{l+s+s1}{\PYZsq{}}\PY{l+s+s1}{serif}\PY{l+s+s1}{\PYZsq{}}\PY{p}{)}
\end{Verbatim}
\end{tcolorbox}

    \begin{tcolorbox}[breakable, size=fbox, boxrule=1pt, pad at break*=1mm,colback=cellbackground, colframe=cellborder]
\prompt{In}{incolor}{25}{\boxspacing}
\begin{Verbatim}[commandchars=\\\{\}]
\PY{n}{src\PYZus{}image\PYZus{}path} \PY{o}{=} \PY{l+s+s2}{\PYZdq{}}\PY{l+s+s2}{roman.jpg}\PY{l+s+s2}{\PYZdq{}}

\PY{k}{def} \PY{n+nf}{read\PYZus{}image}\PY{p}{(}\PY{n}{path}\PY{p}{)}\PY{p}{:}
    \PY{k}{return} \PY{n}{imageio}\PY{o}{.}\PY{n}{imread}\PY{p}{(}\PY{n}{path}\PY{p}{)}

\PY{k}{def} \PY{n+nf}{save\PYZus{}image}\PY{p}{(}\PY{n}{path}\PY{p}{,} \PY{n}{image}\PY{p}{)}\PY{p}{:}
    \PY{n}{imageio}\PY{o}{.}\PY{n}{imwrite}\PY{p}{(}\PY{n}{path}\PY{p}{,} \PY{n}{image}\PY{p}{,} \PY{n+nb}{format}\PY{o}{=}\PY{l+s+s1}{\PYZsq{}}\PY{l+s+s1}{png}\PY{l+s+s1}{\PYZsq{}}\PY{p}{)}

\PY{k}{def} \PY{n+nf}{show\PYZus{}image}\PY{p}{(}\PY{n}{arr}\PY{p}{,} \PY{o}{*}\PY{o}{*}\PY{n}{kwargs}\PY{p}{)}\PY{p}{:}
    \PY{n}{plt}\PY{o}{.}\PY{n}{axis}\PY{p}{(}\PY{l+s+s1}{\PYZsq{}}\PY{l+s+s1}{off}\PY{l+s+s1}{\PYZsq{}}\PY{p}{)}
    \PY{n}{plt}\PY{o}{.}\PY{n}{imshow}\PY{p}{(}\PY{n}{arr}\PY{p}{,} \PY{o}{*}\PY{o}{*}\PY{n}{kwargs}\PY{p}{)}

\PY{n}{src\PYZus{}image} \PY{o}{=} \PY{n}{read\PYZus{}image}\PY{p}{(}\PY{n}{src\PYZus{}image\PYZus{}path}\PY{p}{)}
\PY{n+nb}{print}\PY{p}{(}\PY{l+s+s2}{\PYZdq{}}\PY{l+s+s2}{Image }\PY{l+s+si}{\PYZob{}\PYZcb{}}\PY{l+s+s2}{ loaded: (}\PY{l+s+si}{\PYZob{}\PYZcb{}}\PY{l+s+s2}{ x }\PY{l+s+si}{\PYZob{}\PYZcb{}}\PY{l+s+s2}{)}\PY{l+s+s2}{\PYZdq{}}\PY{o}{.}\PY{n}{format}\PY{p}{(}\PY{n}{src\PYZus{}image\PYZus{}path}\PY{p}{,} \PY{n}{src\PYZus{}image}\PY{o}{.}\PY{n}{shape}\PY{p}{[}\PY{l+m+mi}{1}\PY{p}{]}\PY{p}{,} \PY{n}{src\PYZus{}image}\PY{o}{.}\PY{n}{shape}\PY{p}{[}\PY{l+m+mi}{0}\PY{p}{]}\PY{p}{)}\PY{p}{)}
\PY{n}{show\PYZus{}image}\PY{p}{(}\PY{n}{src\PYZus{}image}\PY{p}{)}
\PY{n}{save\PYZus{}image}\PY{p}{(}\PY{l+s+s2}{\PYZdq{}}\PY{l+s+s2}{images/src.png}\PY{l+s+s2}{\PYZdq{}}\PY{p}{,} \PY{n}{src\PYZus{}image}\PY{p}{)}
\end{Verbatim}
\end{tcolorbox}

    \begin{Verbatim}[commandchars=\\\{\}]
Image roman.jpg loaded: (1440 x 900)
    \end{Verbatim}

    \begin{center}
    \adjustimage{max size={0.9\linewidth}{0.9\paperheight}}{main_files/main_3_1.png}
    \end{center}
    { \hspace*{\fill} \\}
    
    \hypertarget{problem-1}{%
\section{Problem 1}\label{problem-1}}

\hypertarget{generate-grayscale-image-from-the-r-channel}{%
\subsection{1.1 Generate grayscale image from the R
channel}\label{generate-grayscale-image-from-the-r-channel}}

    \begin{tcolorbox}[breakable, size=fbox, boxrule=1pt, pad at break*=1mm,colback=cellbackground, colframe=cellborder]
\prompt{In}{incolor}{27}{\boxspacing}
\begin{Verbatim}[commandchars=\\\{\}]
\PY{n}{src\PYZus{}red} \PY{o}{=} \PY{n}{src\PYZus{}image}\PY{p}{[}\PY{p}{:}\PY{p}{,} \PY{p}{:}\PY{p}{,} \PY{l+m+mi}{0}\PY{p}{]}
\PY{n}{show\PYZus{}image}\PY{p}{(}\PY{n}{src\PYZus{}red}\PY{p}{,} \PY{n}{cmap}\PY{o}{=}\PY{l+s+s1}{\PYZsq{}}\PY{l+s+s1}{gray}\PY{l+s+s1}{\PYZsq{}}\PY{p}{)}
\PY{n}{save\PYZus{}image}\PY{p}{(}\PY{l+s+s2}{\PYZdq{}}\PY{l+s+s2}{images/src\PYZus{}R\PYZus{}grayscale.png}\PY{l+s+s2}{\PYZdq{}}\PY{p}{,} \PY{n}{src\PYZus{}red}\PY{p}{)}
\end{Verbatim}
\end{tcolorbox}

    \begin{center}
    \adjustimage{max size={0.9\linewidth}{0.9\paperheight}}{main_files/main_5_0.png}
    \end{center}
    { \hspace*{\fill} \\}
    
    \hypertarget{enhance-the-image-with-histogram-equalization}{%
\subsection{1.2 Enhance the image with histogram
equalization}\label{enhance-the-image-with-histogram-equalization}}

We will use the \texttt{equalize\_hist} function from the
\texttt{skimage} module to enhance the grayscale image.

    \begin{tcolorbox}[breakable, size=fbox, boxrule=1pt, pad at break*=1mm,colback=cellbackground, colframe=cellborder]
\prompt{In}{incolor}{86}{\boxspacing}
\begin{Verbatim}[commandchars=\\\{\}]
\PY{n}{eq\PYZus{}red} \PY{o}{=} \PY{n}{exposure}\PY{o}{.}\PY{n}{equalize\PYZus{}hist}\PY{p}{(}\PY{n}{src\PYZus{}red}\PY{p}{)}
\PY{n}{save\PYZus{}image}\PY{p}{(}\PY{l+s+s2}{\PYZdq{}}\PY{l+s+s2}{images/src\PYZus{}R\PYZus{}eq.png}\PY{l+s+s2}{\PYZdq{}}\PY{p}{,} \PY{n}{eq\PYZus{}red}\PY{p}{)}

\PY{c+c1}{\PYZsh{} Show the histograms of both images}
\PY{n}{plt}\PY{o}{.}\PY{n}{figure}\PY{p}{(}\PY{n}{dpi}\PY{o}{=}\PY{l+m+mi}{220}\PY{p}{,} \PY{n}{figsize}\PY{o}{=}\PY{p}{(}\PY{l+m+mi}{10}\PY{p}{,} \PY{l+m+mi}{4}\PY{p}{)}\PY{p}{)}
\PY{n}{plt}\PY{o}{.}\PY{n}{subplot}\PY{p}{(}\PY{l+m+mi}{1}\PY{p}{,} \PY{l+m+mi}{2}\PY{p}{,} \PY{l+m+mi}{1}\PY{p}{)}
\PY{n}{plt}\PY{o}{.}\PY{n}{hist}\PY{p}{(}\PY{n}{src\PYZus{}red}\PY{o}{.}\PY{n}{flatten}\PY{p}{(}\PY{p}{)}\PY{p}{,} \PY{n}{bins}\PY{o}{=}\PY{l+m+mi}{256}\PY{p}{,} \PY{n+nb}{range}\PY{o}{=}\PY{p}{(}\PY{l+m+mi}{0}\PY{p}{,} \PY{l+m+mi}{255}\PY{p}{)}\PY{p}{,} \PY{n}{ec}\PY{o}{=}\PY{l+s+s1}{\PYZsq{}}\PY{l+s+s1}{red}\PY{l+s+s1}{\PYZsq{}}\PY{p}{,} \PY{n}{fc}\PY{o}{=}\PY{l+s+s1}{\PYZsq{}}\PY{l+s+s1}{red}\PY{l+s+s1}{\PYZsq{}}\PY{p}{)}
\PY{n}{plt}\PY{o}{.}\PY{n}{xlim}\PY{p}{(}\PY{l+m+mi}{0}\PY{p}{,} \PY{l+m+mi}{255}\PY{p}{)}
\PY{n}{plt}\PY{o}{.}\PY{n}{title}\PY{p}{(}\PY{l+s+s1}{\PYZsq{}}\PY{l+s+s1}{Source image histogram (R channel)}\PY{l+s+s1}{\PYZsq{}}\PY{p}{,} \PY{n}{loc}\PY{o}{=}\PY{l+s+s2}{\PYZdq{}}\PY{l+s+s2}{center}\PY{l+s+s2}{\PYZdq{}}\PY{p}{)}
\PY{n}{plt}\PY{o}{.}\PY{n}{subplot}\PY{p}{(}\PY{l+m+mi}{1}\PY{p}{,} \PY{l+m+mi}{2}\PY{p}{,} \PY{l+m+mi}{2}\PY{p}{)}
\PY{n}{plt}\PY{o}{.}\PY{n}{hist}\PY{p}{(}\PY{n}{eq\PYZus{}red}\PY{o}{.}\PY{n}{ravel}\PY{p}{(}\PY{p}{)}\PY{p}{,} \PY{n}{bins}\PY{o}{=}\PY{l+m+mi}{256}\PY{p}{,} \PY{n+nb}{range}\PY{o}{=}\PY{p}{(}\PY{l+m+mf}{0.0}\PY{p}{,} \PY{l+m+mf}{1.0}\PY{p}{)}\PY{p}{,} \PY{n}{ec}\PY{o}{=}\PY{l+s+s1}{\PYZsq{}}\PY{l+s+s1}{k}\PY{l+s+s1}{\PYZsq{}}\PY{p}{,} \PY{n}{fc}\PY{o}{=}\PY{l+s+s1}{\PYZsq{}}\PY{l+s+s1}{k}\PY{l+s+s1}{\PYZsq{}}\PY{p}{)}
\PY{n}{plt}\PY{o}{.}\PY{n}{xlim}\PY{p}{(}\PY{p}{[}\PY{l+m+mf}{0.0}\PY{p}{,} \PY{l+m+mf}{1.0}\PY{p}{]}\PY{p}{)}
\PY{n}{plt}\PY{o}{.}\PY{n}{title}\PY{p}{(}\PY{l+s+s1}{\PYZsq{}}\PY{l+s+s1}{Equalized image histogram}\PY{l+s+s1}{\PYZsq{}}\PY{p}{,} \PY{n}{loc}\PY{o}{=}\PY{l+s+s2}{\PYZdq{}}\PY{l+s+s2}{center}\PY{l+s+s2}{\PYZdq{}}\PY{p}{)}
\PY{n}{plt}\PY{o}{.}\PY{n}{subplots\PYZus{}adjust}\PY{p}{(}\PY{n}{wspace}\PY{o}{=}\PY{l+m+mf}{0.2}\PY{p}{)}
\PY{n}{plt}\PY{o}{.}\PY{n}{show}\PY{p}{(}\PY{p}{)}
\end{Verbatim}
\end{tcolorbox}

    \begin{Verbatim}[commandchars=\\\{\}]
Lossy conversion from float64 to uint8. Range [0, 1]. Convert image to uint8
prior to saving to suppress this warning.
    \end{Verbatim}

    \begin{center}
    \adjustimage{max size={0.9\linewidth}{0.9\paperheight}}{main_files/main_7_1.png}
    \end{center}
    { \hspace*{\fill} \\}
    
    We see from the above image that the histogram spreads out over the
entire range of pixel values in the equalized histogram, hence we affirm
that the image output is working as expected.

    \hypertarget{problem-2}{%
\section{Problem 2}\label{problem-2}}

\hypertarget{use-exponential-distribution-to-match-the-histogram-of-the-image}{%
\subsection{2.1 Use exponential distribution to match the histogram of
the
image}\label{use-exponential-distribution-to-match-the-histogram-of-the-image}}

Here we will use the \texttt{match\_histograms} function from the
\texttt{skimage} module to match the histogram of the image.

    \begin{tcolorbox}[breakable, size=fbox, boxrule=1pt, pad at break*=1mm,colback=cellbackground, colframe=cellborder]
\prompt{In}{incolor}{77}{\boxspacing}
\begin{Verbatim}[commandchars=\\\{\}]
\PY{c+c1}{\PYZsh{} Create the exponential distribution of identical size}
\PY{n}{exp\PYZus{}distribution} \PY{o}{=} \PY{n}{np}\PY{o}{.}\PY{n}{random}\PY{o}{.}\PY{n}{exponential}\PY{p}{(}\PY{n}{size}\PY{o}{=}\PY{n}{src\PYZus{}red}\PY{o}{.}\PY{n}{shape}\PY{p}{)}
\PY{c+c1}{\PYZsh{} Match the histogram of the image}
\PY{n}{exp\PYZus{}matched\PYZus{}red} \PY{o}{=} \PY{n}{exposure}\PY{o}{.}\PY{n}{match\PYZus{}histograms}\PY{p}{(}\PY{n}{src\PYZus{}red}\PY{p}{,} \PY{n}{exp\PYZus{}distribution}\PY{p}{)}
\PY{c+c1}{\PYZsh{} Show the result}
\PY{n}{show\PYZus{}image}\PY{p}{(}\PY{n}{exp\PYZus{}matched\PYZus{}red}\PY{p}{,} \PY{n}{cmap}\PY{o}{=}\PY{l+s+s1}{\PYZsq{}}\PY{l+s+s1}{gray}\PY{l+s+s1}{\PYZsq{}}\PY{p}{)}
\PY{n}{save\PYZus{}image}\PY{p}{(}\PY{l+s+s2}{\PYZdq{}}\PY{l+s+s2}{images/src\PYZus{}R\PYZus{}matched\PYZus{}exp.png}\PY{l+s+s2}{\PYZdq{}}\PY{p}{,} \PY{n}{exp\PYZus{}matched\PYZus{}red}\PY{p}{)}
\end{Verbatim}
\end{tcolorbox}

    \begin{Verbatim}[commandchars=\\\{\}]
Lossy conversion from float64 to uint8. Range [0.047314245281897226,
15.798753446298184]. Convert image to uint8 prior to saving to suppress this
warning.
    \end{Verbatim}

    \begin{center}
    \adjustimage{max size={0.9\linewidth}{0.9\paperheight}}{main_files/main_10_1.png}
    \end{center}
    { \hspace*{\fill} \\}
    
    We see that the image is darker than the original image. Similarly,
we'll compare the histogram of the image with the histogram of the
original image.

    \begin{tcolorbox}[breakable, size=fbox, boxrule=1pt, pad at break*=1mm,colback=cellbackground, colframe=cellborder]
\prompt{In}{incolor}{101}{\boxspacing}
\begin{Verbatim}[commandchars=\\\{\}]
\PY{c+c1}{\PYZsh{} Show the histograms of both images}
\PY{n}{plt}\PY{o}{.}\PY{n}{figure}\PY{p}{(}\PY{n}{dpi}\PY{o}{=}\PY{l+m+mi}{220}\PY{p}{,} \PY{n}{figsize}\PY{o}{=}\PY{p}{(}\PY{l+m+mi}{10}\PY{p}{,} \PY{l+m+mi}{4}\PY{p}{)}\PY{p}{)}
\PY{n}{plt}\PY{o}{.}\PY{n}{subplot}\PY{p}{(}\PY{l+m+mi}{1}\PY{p}{,} \PY{l+m+mi}{2}\PY{p}{,} \PY{l+m+mi}{1}\PY{p}{)}
\PY{n}{plt}\PY{o}{.}\PY{n}{hist}\PY{p}{(}\PY{n}{src\PYZus{}red}\PY{o}{.}\PY{n}{flatten}\PY{p}{(}\PY{p}{)}\PY{p}{,} \PY{n}{bins}\PY{o}{=}\PY{l+m+mi}{256}\PY{p}{,} \PY{n+nb}{range}\PY{o}{=}\PY{p}{(}\PY{l+m+mi}{0}\PY{p}{,} \PY{l+m+mi}{255}\PY{p}{)}\PY{p}{,} \PY{n}{ec}\PY{o}{=}\PY{l+s+s1}{\PYZsq{}}\PY{l+s+s1}{red}\PY{l+s+s1}{\PYZsq{}}\PY{p}{,} \PY{n}{fc}\PY{o}{=}\PY{l+s+s1}{\PYZsq{}}\PY{l+s+s1}{red}\PY{l+s+s1}{\PYZsq{}}\PY{p}{)}
\PY{n}{plt}\PY{o}{.}\PY{n}{xlim}\PY{p}{(}\PY{l+m+mi}{0}\PY{p}{,} \PY{l+m+mi}{255}\PY{p}{)}
\PY{n}{plt}\PY{o}{.}\PY{n}{title}\PY{p}{(}\PY{l+s+s1}{\PYZsq{}}\PY{l+s+s1}{Source image histogram (R channel)}\PY{l+s+s1}{\PYZsq{}}\PY{p}{,} \PY{n}{loc}\PY{o}{=}\PY{l+s+s2}{\PYZdq{}}\PY{l+s+s2}{center}\PY{l+s+s2}{\PYZdq{}}\PY{p}{)}
\PY{n}{plt}\PY{o}{.}\PY{n}{subplot}\PY{p}{(}\PY{l+m+mi}{1}\PY{p}{,} \PY{l+m+mi}{2}\PY{p}{,} \PY{l+m+mi}{2}\PY{p}{)}
\PY{n}{plt}\PY{o}{.}\PY{n}{hist}\PY{p}{(}\PY{n}{exp\PYZus{}matched\PYZus{}red}\PY{o}{.}\PY{n}{flatten}\PY{p}{(}\PY{p}{)}\PY{p}{,} \PY{n}{bins}\PY{o}{=}\PY{l+m+mi}{256}\PY{p}{,} \PY{n+nb}{range}\PY{o}{=}\PY{p}{(}\PY{l+m+mf}{0.0}\PY{p}{,} \PY{l+m+mf}{8.0}\PY{p}{)}\PY{p}{,} \PY{n}{ec}\PY{o}{=}\PY{l+s+s1}{\PYZsq{}}\PY{l+s+s1}{k}\PY{l+s+s1}{\PYZsq{}}\PY{p}{,} \PY{n}{fc}\PY{o}{=}\PY{l+s+s1}{\PYZsq{}}\PY{l+s+s1}{k}\PY{l+s+s1}{\PYZsq{}}\PY{p}{)}
\PY{c+c1}{\PYZsh{} Line\PYZhy{}style hist for exp distribution}
\PY{n}{n}\PY{p}{,} \PY{n}{x}\PY{p}{,} \PY{n}{\PYZus{}} \PY{o}{=} \PY{n}{plt}\PY{o}{.}\PY{n}{hist}\PY{p}{(}\PY{n}{exp\PYZus{}distribution}\PY{o}{.}\PY{n}{flatten}\PY{p}{(}\PY{p}{)}\PY{p}{,} \PY{n}{bins}\PY{o}{=}\PY{l+m+mi}{256}\PY{p}{,} \PY{n+nb}{range}\PY{o}{=}\PY{p}{(}\PY{l+m+mf}{0.0}\PY{p}{,} \PY{l+m+mf}{8.0}\PY{p}{)}\PY{p}{,} \PY{n}{ec}\PY{o}{=}\PY{l+s+s1}{\PYZsq{}}\PY{l+s+s1}{r}\PY{l+s+s1}{\PYZsq{}}\PY{p}{,} \PY{n}{fc}\PY{o}{=}\PY{l+s+s1}{\PYZsq{}}\PY{l+s+s1}{r}\PY{l+s+s1}{\PYZsq{}}\PY{p}{,} \PY{n}{alpha}\PY{o}{=}\PY{l+m+mi}{0}\PY{p}{)}
\PY{n}{plt}\PY{o}{.}\PY{n}{plot}\PY{p}{(}\PY{n}{x}\PY{p}{[}\PY{p}{:}\PY{o}{\PYZhy{}}\PY{l+m+mi}{1}\PY{p}{]}\PY{p}{,} \PY{n}{n}\PY{p}{,} \PY{l+s+s1}{\PYZsq{}}\PY{l+s+s1}{r\PYZhy{}\PYZhy{}}\PY{l+s+s1}{\PYZsq{}}\PY{p}{,} \PY{n}{linewidth}\PY{o}{=}\PY{l+m+mi}{1}\PY{p}{)}
\PY{n}{plt}\PY{o}{.}\PY{n}{legend}\PY{p}{(}\PY{p}{[}\PY{l+s+s1}{\PYZsq{}}\PY{l+s+s1}{Exponential distribution}\PY{l+s+s1}{\PYZsq{}}\PY{p}{,} \PY{l+s+s1}{\PYZsq{}}\PY{l+s+s1}{Image histogram}\PY{l+s+s1}{\PYZsq{}}\PY{p}{]}\PY{p}{,} \PY{n}{loc}\PY{o}{=}\PY{l+s+s2}{\PYZdq{}}\PY{l+s+s2}{upper right}\PY{l+s+s2}{\PYZdq{}}\PY{p}{)}
\PY{n}{plt}\PY{o}{.}\PY{n}{xlim}\PY{p}{(}\PY{p}{[}\PY{l+m+mf}{0.0}\PY{p}{,} \PY{l+m+mf}{8.0}\PY{p}{]}\PY{p}{)}
\PY{n}{plt}\PY{o}{.}\PY{n}{title}\PY{p}{(}\PY{l+s+s1}{\PYZsq{}}\PY{l+s+s1}{Equalized image histogram}\PY{l+s+s1}{\PYZsq{}}\PY{p}{,} \PY{n}{loc}\PY{o}{=}\PY{l+s+s2}{\PYZdq{}}\PY{l+s+s2}{center}\PY{l+s+s2}{\PYZdq{}}\PY{p}{)}
\PY{n}{plt}\PY{o}{.}\PY{n}{subplots\PYZus{}adjust}\PY{p}{(}\PY{n}{wspace}\PY{o}{=}\PY{l+m+mf}{0.2}\PY{p}{)}
\PY{n}{plt}\PY{o}{.}\PY{n}{show}\PY{p}{(}\PY{p}{)}
\end{Verbatim}
\end{tcolorbox}

    \begin{center}
    \adjustimage{max size={0.9\linewidth}{0.9\paperheight}}{main_files/main_12_0.png}
    \end{center}
    { \hspace*{\fill} \\}
    
    We see that the histogram of the processed image is much alike to
exponenetial distribution, hence we affirm that the image output is
working as expected.

    \hypertarget{use-gaussian-distribution-to-match-the-histogram-of-the-image}{%
\subsection{2.2 Use Gaussian distribution to match the histogram of the
image}\label{use-gaussian-distribution-to-match-the-histogram-of-the-image}}

    \begin{tcolorbox}[breakable, size=fbox, boxrule=1pt, pad at break*=1mm,colback=cellbackground, colframe=cellborder]
\prompt{In}{incolor}{100}{\boxspacing}
\begin{Verbatim}[commandchars=\\\{\}]
\PY{c+c1}{\PYZsh{} Create the Gaussian distribution of identical size}
\PY{n}{gaussian\PYZus{}distribution} \PY{o}{=} \PY{n}{np}\PY{o}{.}\PY{n}{random}\PY{o}{.}\PY{n}{normal}\PY{p}{(}\PY{n}{size}\PY{o}{=}\PY{n}{src\PYZus{}red}\PY{o}{.}\PY{n}{shape}\PY{p}{)}
\PY{c+c1}{\PYZsh{} Match the histogram of the image}
\PY{n}{gaussian\PYZus{}matched\PYZus{}red} \PY{o}{=} \PY{n}{exposure}\PY{o}{.}\PY{n}{match\PYZus{}histograms}\PY{p}{(}\PY{n}{src\PYZus{}red}\PY{p}{,} \PY{n}{gaussian\PYZus{}distribution}\PY{p}{)}
\PY{c+c1}{\PYZsh{} Show the result}
\PY{n}{show\PYZus{}image}\PY{p}{(}\PY{n}{gaussian\PYZus{}matched\PYZus{}red}\PY{p}{,} \PY{n}{cmap}\PY{o}{=}\PY{l+s+s1}{\PYZsq{}}\PY{l+s+s1}{gray}\PY{l+s+s1}{\PYZsq{}}\PY{p}{)}
\PY{n}{save\PYZus{}image}\PY{p}{(}\PY{l+s+s2}{\PYZdq{}}\PY{l+s+s2}{images/src\PYZus{}R\PYZus{}matched\PYZus{}gaussian.png}\PY{l+s+s2}{\PYZdq{}}\PY{p}{,} \PY{n}{gaussian\PYZus{}matched\PYZus{}red}\PY{p}{)}
\end{Verbatim}
\end{tcolorbox}

    \begin{Verbatim}[commandchars=\\\{\}]
Lossy conversion from float64 to uint8. Range [-1.6847300761121298,
5.190763058438374]. Convert image to uint8 prior to saving to suppress this
warning.
    \end{Verbatim}

    \begin{center}
    \adjustimage{max size={0.9\linewidth}{0.9\paperheight}}{main_files/main_15_1.png}
    \end{center}
    { \hspace*{\fill} \\}
    
    \begin{tcolorbox}[breakable, size=fbox, boxrule=1pt, pad at break*=1mm,colback=cellbackground, colframe=cellborder]
\prompt{In}{incolor}{102}{\boxspacing}
\begin{Verbatim}[commandchars=\\\{\}]
\PY{c+c1}{\PYZsh{} Show the histograms of both images}
\PY{n}{plt}\PY{o}{.}\PY{n}{figure}\PY{p}{(}\PY{n}{dpi}\PY{o}{=}\PY{l+m+mi}{220}\PY{p}{,} \PY{n}{figsize}\PY{o}{=}\PY{p}{(}\PY{l+m+mi}{10}\PY{p}{,} \PY{l+m+mi}{4}\PY{p}{)}\PY{p}{)}
\PY{n}{plt}\PY{o}{.}\PY{n}{subplot}\PY{p}{(}\PY{l+m+mi}{1}\PY{p}{,} \PY{l+m+mi}{2}\PY{p}{,} \PY{l+m+mi}{1}\PY{p}{)}
\PY{n}{plt}\PY{o}{.}\PY{n}{hist}\PY{p}{(}\PY{n}{src\PYZus{}red}\PY{o}{.}\PY{n}{flatten}\PY{p}{(}\PY{p}{)}\PY{p}{,} \PY{n}{bins}\PY{o}{=}\PY{l+m+mi}{256}\PY{p}{,} \PY{n+nb}{range}\PY{o}{=}\PY{p}{(}\PY{l+m+mi}{0}\PY{p}{,} \PY{l+m+mi}{255}\PY{p}{)}\PY{p}{,} \PY{n}{ec}\PY{o}{=}\PY{l+s+s1}{\PYZsq{}}\PY{l+s+s1}{red}\PY{l+s+s1}{\PYZsq{}}\PY{p}{,} \PY{n}{fc}\PY{o}{=}\PY{l+s+s1}{\PYZsq{}}\PY{l+s+s1}{red}\PY{l+s+s1}{\PYZsq{}}\PY{p}{)}
\PY{n}{plt}\PY{o}{.}\PY{n}{xlim}\PY{p}{(}\PY{l+m+mi}{0}\PY{p}{,} \PY{l+m+mi}{255}\PY{p}{)}
\PY{n}{plt}\PY{o}{.}\PY{n}{title}\PY{p}{(}\PY{l+s+s1}{\PYZsq{}}\PY{l+s+s1}{Source image histogram (R channel)}\PY{l+s+s1}{\PYZsq{}}\PY{p}{,} \PY{n}{loc}\PY{o}{=}\PY{l+s+s2}{\PYZdq{}}\PY{l+s+s2}{center}\PY{l+s+s2}{\PYZdq{}}\PY{p}{)}
\PY{n}{plt}\PY{o}{.}\PY{n}{subplot}\PY{p}{(}\PY{l+m+mi}{1}\PY{p}{,} \PY{l+m+mi}{2}\PY{p}{,} \PY{l+m+mi}{2}\PY{p}{)}
\PY{n}{plt}\PY{o}{.}\PY{n}{hist}\PY{p}{(}\PY{n}{gaussian\PYZus{}matched\PYZus{}red}\PY{o}{.}\PY{n}{flatten}\PY{p}{(}\PY{p}{)}\PY{p}{,} \PY{n}{bins}\PY{o}{=}\PY{l+m+mi}{256}\PY{p}{,} \PY{n+nb}{range}\PY{o}{=}\PY{p}{(}\PY{l+m+mf}{0.0}\PY{p}{,} \PY{l+m+mf}{8.0}\PY{p}{)}\PY{p}{,} \PY{n}{ec}\PY{o}{=}\PY{l+s+s1}{\PYZsq{}}\PY{l+s+s1}{k}\PY{l+s+s1}{\PYZsq{}}\PY{p}{,} \PY{n}{fc}\PY{o}{=}\PY{l+s+s1}{\PYZsq{}}\PY{l+s+s1}{k}\PY{l+s+s1}{\PYZsq{}}\PY{p}{)}
\PY{c+c1}{\PYZsh{} Line\PYZhy{}style hist for gaussian distribution}
\PY{n}{n}\PY{p}{,} \PY{n}{x}\PY{p}{,} \PY{n}{\PYZus{}} \PY{o}{=} \PY{n}{plt}\PY{o}{.}\PY{n}{hist}\PY{p}{(}\PY{n}{gaussian\PYZus{}distribution}\PY{o}{.}\PY{n}{flatten}\PY{p}{(}\PY{p}{)}\PY{p}{,} \PY{n}{bins}\PY{o}{=}\PY{l+m+mi}{256}\PY{p}{,} \PY{n+nb}{range}\PY{o}{=}\PY{p}{(}\PY{l+m+mf}{0.0}\PY{p}{,} \PY{l+m+mf}{8.0}\PY{p}{)}\PY{p}{,} \PY{n}{ec}\PY{o}{=}\PY{l+s+s1}{\PYZsq{}}\PY{l+s+s1}{r}\PY{l+s+s1}{\PYZsq{}}\PY{p}{,} \PY{n}{fc}\PY{o}{=}\PY{l+s+s1}{\PYZsq{}}\PY{l+s+s1}{r}\PY{l+s+s1}{\PYZsq{}}\PY{p}{,} \PY{n}{alpha}\PY{o}{=}\PY{l+m+mi}{0}\PY{p}{)}
\PY{n}{plt}\PY{o}{.}\PY{n}{plot}\PY{p}{(}\PY{n}{x}\PY{p}{[}\PY{p}{:}\PY{o}{\PYZhy{}}\PY{l+m+mi}{1}\PY{p}{]}\PY{p}{,} \PY{n}{n}\PY{p}{,} \PY{l+s+s1}{\PYZsq{}}\PY{l+s+s1}{r\PYZhy{}\PYZhy{}}\PY{l+s+s1}{\PYZsq{}}\PY{p}{,} \PY{n}{linewidth}\PY{o}{=}\PY{l+m+mi}{1}\PY{p}{)}
\PY{n}{plt}\PY{o}{.}\PY{n}{legend}\PY{p}{(}\PY{p}{[}\PY{l+s+s1}{\PYZsq{}}\PY{l+s+s1}{Gaussian distribution}\PY{l+s+s1}{\PYZsq{}}\PY{p}{,} \PY{l+s+s1}{\PYZsq{}}\PY{l+s+s1}{Image histogram}\PY{l+s+s1}{\PYZsq{}}\PY{p}{]}\PY{p}{,} \PY{n}{loc}\PY{o}{=}\PY{l+s+s2}{\PYZdq{}}\PY{l+s+s2}{upper right}\PY{l+s+s2}{\PYZdq{}}\PY{p}{)}
\PY{n}{plt}\PY{o}{.}\PY{n}{xlim}\PY{p}{(}\PY{p}{[}\PY{l+m+mf}{0.0}\PY{p}{,} \PY{l+m+mf}{8.0}\PY{p}{]}\PY{p}{)}
\PY{n}{plt}\PY{o}{.}\PY{n}{title}\PY{p}{(}\PY{l+s+s1}{\PYZsq{}}\PY{l+s+s1}{Equalized image histogram}\PY{l+s+s1}{\PYZsq{}}\PY{p}{,} \PY{n}{loc}\PY{o}{=}\PY{l+s+s2}{\PYZdq{}}\PY{l+s+s2}{center}\PY{l+s+s2}{\PYZdq{}}\PY{p}{)}
\PY{n}{plt}\PY{o}{.}\PY{n}{subplots\PYZus{}adjust}\PY{p}{(}\PY{n}{wspace}\PY{o}{=}\PY{l+m+mf}{0.2}\PY{p}{)}
\PY{n}{plt}\PY{o}{.}\PY{n}{show}\PY{p}{(}\PY{p}{)}
\end{Verbatim}
\end{tcolorbox}

    \begin{center}
    \adjustimage{max size={0.9\linewidth}{0.9\paperheight}}{main_files/main_16_0.png}
    \end{center}
    { \hspace*{\fill} \\}
    
    The result is also as expected.

    \hypertarget{problem-3-matching-the-histogram-of-the-image-with-a-custom-distribution}{%
\section{Problem 3: Matching the histogram of the image with a custom
distribution}\label{problem-3-matching-the-histogram-of-the-image-with-a-custom-distribution}}

Histogram matching is a very common task in image processing. Here we
will use a special distribution to match the histogram of the image:
Gamma distribution.

\[
\begin{align}
f(x;\alpha,\beta) & = \frac{ x^{\alpha-1} e^{-\beta x} \beta^\alpha}{\Gamma(\alpha)} \quad \text{ for } x > 0 \quad \alpha, \beta > 0, \\[6pt]
\Gamma(\alpha) &= (\alpha - 1)! \quad \text{ for } \alpha > 0.
\end{align}
\]

It's such a fun thing to see how the image evolves over different choice
of \(\alpha\) and \(\beta\). Therefore I made a matrix for them!

    

    \begin{tcolorbox}[breakable, size=fbox, boxrule=1pt, pad at break*=1mm,colback=cellbackground, colframe=cellborder]
\prompt{In}{incolor}{116}{\boxspacing}
\begin{Verbatim}[commandchars=\\\{\}]
\PY{n}{alpha\PYZus{}choices} \PY{o}{=} \PY{p}{[}\PY{l+m+mi}{1}\PY{p}{,} \PY{l+m+mi}{2}\PY{p}{,} \PY{l+m+mi}{4}\PY{p}{,} \PY{l+m+mi}{8}\PY{p}{]}
\PY{n}{beta\PYZus{}choices} \PY{o}{=} \PY{p}{[}\PY{l+m+mi}{1}\PY{p}{,} \PY{l+m+mi}{2}\PY{p}{,} \PY{l+m+mi}{4}\PY{p}{,} \PY{l+m+mi}{8}\PY{p}{]}
\PY{n}{gamma\PYZus{}matched\PYZus{}output} \PY{o}{=} \PY{p}{\PYZob{}}\PY{p}{\PYZcb{}}

\PY{n}{plt}\PY{o}{.}\PY{n}{figure}\PY{p}{(}\PY{n}{dpi}\PY{o}{=}\PY{l+m+mi}{220}\PY{p}{,} \PY{n}{figsize}\PY{o}{=}\PY{p}{(}\PY{l+m+mi}{20}\PY{p}{,} \PY{l+m+mi}{20}\PY{p}{)}\PY{p}{)}
\PY{k}{for} \PY{n}{alpha} \PY{o+ow}{in} \PY{n}{alpha\PYZus{}choices}\PY{p}{:}
    \PY{k}{for} \PY{n}{beta} \PY{o+ow}{in} \PY{n}{beta\PYZus{}choices}\PY{p}{:}
        \PY{c+c1}{\PYZsh{} Create the Gamma distribution of identical size}
        \PY{n}{gamma\PYZus{}distribution} \PY{o}{=} \PY{n}{np}\PY{o}{.}\PY{n}{random}\PY{o}{.}\PY{n}{gamma}\PY{p}{(}\PY{n}{alpha}\PY{p}{,} \PY{n}{beta}\PY{p}{,} \PY{n}{size}\PY{o}{=}\PY{n}{src\PYZus{}red}\PY{o}{.}\PY{n}{shape}\PY{p}{)}
        \PY{c+c1}{\PYZsh{} Match the histogram of the image}
        \PY{n}{gamma\PYZus{}matched\PYZus{}red} \PY{o}{=} \PY{n}{exposure}\PY{o}{.}\PY{n}{match\PYZus{}histograms}\PY{p}{(}\PY{n}{src\PYZus{}red}\PY{p}{,} \PY{n}{gamma\PYZus{}distribution}\PY{p}{)}
        \PY{c+c1}{\PYZsh{} Append}
        \PY{n}{gamma\PYZus{}matched\PYZus{}output}\PY{p}{[}\PY{p}{(}\PY{n}{alpha}\PY{p}{,} \PY{n}{beta}\PY{p}{)}\PY{p}{]} \PY{o}{=} \PY{n}{gamma\PYZus{}matched\PYZus{}red}
        \PY{c+c1}{\PYZsh{} Show the result}
        \PY{n}{plt}\PY{o}{.}\PY{n}{subplot}\PY{p}{(}\PY{n+nb}{len}\PY{p}{(}\PY{n}{alpha\PYZus{}choices}\PY{p}{)}\PY{p}{,} \PY{n+nb}{len}\PY{p}{(}\PY{n}{beta\PYZus{}choices}\PY{p}{)}\PY{p}{,} \PY{n}{alpha\PYZus{}choices}\PY{o}{.}\PY{n}{index}\PY{p}{(}\PY{n}{alpha}\PY{p}{)} \PY{o}{*} \PY{n+nb}{len}\PY{p}{(}\PY{n}{beta\PYZus{}choices}\PY{p}{)} \PY{o}{+} \PY{n}{beta\PYZus{}choices}\PY{o}{.}\PY{n}{index}\PY{p}{(}\PY{n}{beta}\PY{p}{)} \PY{o}{+} \PY{l+m+mi}{1}\PY{p}{)}
        \PY{n}{plt}\PY{o}{.}\PY{n}{hist}\PY{p}{(}\PY{n}{gamma\PYZus{}matched\PYZus{}red}\PY{o}{.}\PY{n}{flatten}\PY{p}{(}\PY{p}{)}\PY{p}{,} \PY{n}{bins}\PY{o}{=}\PY{l+m+mi}{256}\PY{p}{,} \PY{n}{ec}\PY{o}{=}\PY{l+s+s1}{\PYZsq{}}\PY{l+s+s1}{k}\PY{l+s+s1}{\PYZsq{}}\PY{p}{,} \PY{n}{fc}\PY{o}{=}\PY{l+s+s1}{\PYZsq{}}\PY{l+s+s1}{k}\PY{l+s+s1}{\PYZsq{}}\PY{p}{)}
        \PY{c+c1}{\PYZsh{} Line\PYZhy{}style hist for gamma distribution}
        \PY{n}{n}\PY{p}{,} \PY{n}{x}\PY{p}{,} \PY{n}{\PYZus{}} \PY{o}{=} \PY{n}{plt}\PY{o}{.}\PY{n}{hist}\PY{p}{(}\PY{n}{gamma\PYZus{}distribution}\PY{o}{.}\PY{n}{flatten}\PY{p}{(}\PY{p}{)}\PY{p}{,} \PY{n}{bins}\PY{o}{=}\PY{l+m+mi}{256}\PY{p}{,} \PY{n}{ec}\PY{o}{=}\PY{l+s+s1}{\PYZsq{}}\PY{l+s+s1}{r}\PY{l+s+s1}{\PYZsq{}}\PY{p}{,} \PY{n}{fc}\PY{o}{=}\PY{l+s+s1}{\PYZsq{}}\PY{l+s+s1}{r}\PY{l+s+s1}{\PYZsq{}}\PY{p}{,} \PY{n}{alpha}\PY{o}{=}\PY{l+m+mi}{0}\PY{p}{)}
        \PY{n}{plt}\PY{o}{.}\PY{n}{plot}\PY{p}{(}\PY{n}{x}\PY{p}{[}\PY{p}{:}\PY{o}{\PYZhy{}}\PY{l+m+mi}{1}\PY{p}{]}\PY{p}{,} \PY{n}{n}\PY{p}{,} \PY{l+s+s1}{\PYZsq{}}\PY{l+s+s1}{r\PYZhy{}\PYZhy{}}\PY{l+s+s1}{\PYZsq{}}\PY{p}{,} \PY{n}{linewidth}\PY{o}{=}\PY{l+m+mi}{1}\PY{p}{)}
        \PY{n}{plt}\PY{o}{.}\PY{n}{legend}\PY{p}{(}\PY{p}{[}\PY{l+s+sa}{f}\PY{l+s+s1}{\PYZsq{}}\PY{l+s+s1}{\PYZdl{}}\PY{l+s+s1}{\PYZbs{}}\PY{l+s+s1}{Gamma(}\PY{l+s+si}{\PYZob{}}\PY{n}{alpha}\PY{l+s+si}{\PYZcb{}}\PY{l+s+s1}{, }\PY{l+s+si}{\PYZob{}}\PY{n}{beta}\PY{l+s+si}{\PYZcb{}}\PY{l+s+s1}{)\PYZdl{}}\PY{l+s+s1}{\PYZsq{}}\PY{p}{,} \PY{l+s+s1}{\PYZsq{}}\PY{l+s+s1}{Image histogram}\PY{l+s+s1}{\PYZsq{}}\PY{p}{]}\PY{p}{,} \PY{n}{loc}\PY{o}{=}\PY{l+s+s2}{\PYZdq{}}\PY{l+s+s2}{upper right}\PY{l+s+s2}{\PYZdq{}}\PY{p}{)}
        \PY{n}{save\PYZus{}image}\PY{p}{(}\PY{l+s+s2}{\PYZdq{}}\PY{l+s+s2}{images/gamma/src\PYZus{}R\PYZus{}matched\PYZus{}gamma\PYZus{}}\PY{l+s+si}{\PYZob{}\PYZcb{}}\PY{l+s+s2}{\PYZus{}}\PY{l+s+si}{\PYZob{}\PYZcb{}}\PY{l+s+s2}{.png}\PY{l+s+s2}{\PYZdq{}}\PY{o}{.}\PY{n}{format}\PY{p}{(}\PY{n}{alpha}\PY{p}{,} \PY{n}{beta}\PY{p}{)}\PY{p}{,} \PY{n}{gamma\PYZus{}matched\PYZus{}red}\PY{p}{)}
        
\end{Verbatim}
\end{tcolorbox}

    \begin{Verbatim}[commandchars=\\\{\}]
Lossy conversion from float64 to uint8. Range [0.04686332489032355,
17.132637074077678]. Convert image to uint8 prior to saving to suppress this
warning.
Lossy conversion from float64 to uint8. Range [0.0942010405091525,
28.65813646918947]. Convert image to uint8 prior to saving to suppress this
warning.
Lossy conversion from float64 to uint8. Range [0.18731154699543484,
61.559977313234405]. Convert image to uint8 prior to saving to suppress this
warning.
Lossy conversion from float64 to uint8. Range [0.3789154473435378,
125.60995841414295]. Convert image to uint8 prior to saving to suppress this
warning.
Lossy conversion from float64 to uint8. Range [0.3382678692598206,
19.66257732784066]. Convert image to uint8 prior to saving to suppress this
warning.
Lossy conversion from float64 to uint8. Range [0.6795090875160964,
36.30392817047958]. Convert image to uint8 prior to saving to suppress this
warning.
Lossy conversion from float64 to uint8. Range [1.348683333433539,
71.12107239552121]. Convert image to uint8 prior to saving to suppress this
warning.
Lossy conversion from float64 to uint8. Range [2.70736267695548,
145.19331426427135]. Convert image to uint8 prior to saving to suppress this
warning.
Lossy conversion from float64 to uint8. Range [1.329054235222038,
22.84864125261818]. Convert image to uint8 prior to saving to suppress this
warning.
Lossy conversion from float64 to uint8. Range [2.656430849325686,
46.26725906688678]. Convert image to uint8 prior to saving to suppress this
warning.
Lossy conversion from float64 to uint8. Range [5.305456287631037,
84.24761263659256]. Convert image to uint8 prior to saving to suppress this
warning.
Lossy conversion from float64 to uint8. Range [10.614481245629472,
172.56615643652515]. Convert image to uint8 prior to saving to suppress this
warning.
Lossy conversion from float64 to uint8. Range [3.907252550042045,
29.897231753841577]. Convert image to uint8 prior to saving to suppress this
warning.
Lossy conversion from float64 to uint8. Range [7.8163612674291425,
70.40835568331197]. Convert image to uint8 prior to saving to suppress this
warning.
Lossy conversion from float64 to uint8. Range [15.65236008968683,
115.3744217442024]. Convert image to uint8 prior to saving to suppress this
warning.
Lossy conversion from float64 to uint8. Range [31.283849242764603,
221.49420837321517]. Convert image to uint8 prior to saving to suppress this
warning.
    \end{Verbatim}

    \begin{center}
    \adjustimage{max size={0.9\linewidth}{0.9\paperheight}}{main_files/main_20_1.png}
    \end{center}
    { \hspace*{\fill} \\}
    
    \begin{tcolorbox}[breakable, size=fbox, boxrule=1pt, pad at break*=1mm,colback=cellbackground, colframe=cellborder]
\prompt{In}{incolor}{120}{\boxspacing}
\begin{Verbatim}[commandchars=\\\{\}]
\PY{c+c1}{\PYZsh{} And plot the images}
\PY{n}{plt}\PY{o}{.}\PY{n}{figure}\PY{p}{(}\PY{n}{dpi}\PY{o}{=}\PY{l+m+mi}{220}\PY{p}{,} \PY{n}{figsize}\PY{o}{=}\PY{p}{(}\PY{l+m+mi}{20}\PY{p}{,} \PY{l+m+mi}{20}\PY{p}{)}\PY{p}{)}
\PY{k}{for} \PY{n}{alpha} \PY{o+ow}{in} \PY{n}{alpha\PYZus{}choices}\PY{p}{:}
    \PY{k}{for} \PY{n}{beta} \PY{o+ow}{in} \PY{n}{beta\PYZus{}choices}\PY{p}{:}
        \PY{n}{plt}\PY{o}{.}\PY{n}{subplot}\PY{p}{(}\PY{n+nb}{len}\PY{p}{(}\PY{n}{alpha\PYZus{}choices}\PY{p}{)}\PY{p}{,} \PY{n+nb}{len}\PY{p}{(}\PY{n}{beta\PYZus{}choices}\PY{p}{)}\PY{p}{,} \PY{n}{alpha\PYZus{}choices}\PY{o}{.}\PY{n}{index}\PY{p}{(}\PY{n}{alpha}\PY{p}{)} \PY{o}{*} \PY{n+nb}{len}\PY{p}{(}\PY{n}{beta\PYZus{}choices}\PY{p}{)} \PY{o}{+} \PY{n}{beta\PYZus{}choices}\PY{o}{.}\PY{n}{index}\PY{p}{(}\PY{n}{beta}\PY{p}{)} \PY{o}{+} \PY{l+m+mi}{1}\PY{p}{)}
        \PY{n}{show\PYZus{}image}\PY{p}{(}\PY{n}{gamma\PYZus{}matched\PYZus{}output}\PY{p}{[}\PY{p}{(}\PY{n}{alpha}\PY{p}{,} \PY{n}{beta}\PY{p}{)}\PY{p}{]}\PY{p}{,} \PY{n}{cmap}\PY{o}{=}\PY{l+s+s1}{\PYZsq{}}\PY{l+s+s1}{gray}\PY{l+s+s1}{\PYZsq{}}\PY{p}{)}
        \PY{n}{plt}\PY{o}{.}\PY{n}{title}\PY{p}{(}\PY{l+s+sa}{f}\PY{l+s+s1}{\PYZsq{}}\PY{l+s+s1}{\PYZdl{}}\PY{l+s+s1}{\PYZbs{}}\PY{l+s+s1}{Gamma(}\PY{l+s+si}{\PYZob{}}\PY{n}{alpha}\PY{l+s+si}{\PYZcb{}}\PY{l+s+s1}{, }\PY{l+s+si}{\PYZob{}}\PY{n}{beta}\PY{l+s+si}{\PYZcb{}}\PY{l+s+s1}{)\PYZdl{}}\PY{l+s+s1}{\PYZsq{}}\PY{p}{)}
\PY{c+c1}{\PYZsh{} Adjust the spacing between subplots}
\PY{n}{plt}\PY{o}{.}\PY{n}{subplots\PYZus{}adjust}\PY{p}{(}\PY{n}{wspace}\PY{o}{=}\PY{l+m+mf}{0.1}\PY{p}{)}
\PY{n}{plt}\PY{o}{.}\PY{n}{subplots\PYZus{}adjust}\PY{p}{(}\PY{n}{hspace}\PY{o}{=}\PY{o}{\PYZhy{}}\PY{l+m+mf}{0.4}\PY{p}{)}
\PY{n}{plt}\PY{o}{.}\PY{n}{show}\PY{p}{(}\PY{p}{)}
\end{Verbatim}
\end{tcolorbox}

    \begin{center}
    \adjustimage{max size={0.9\linewidth}{0.9\paperheight}}{main_files/main_21_0.png}
    \end{center}
    { \hspace*{\fill} \\}
    
    We see that with bigger choice of \(\alpha\) and \(\beta\), the image
becomes brighter. This can be seen from the histograms, as the
distribution moves slowly onwards to the right, the histogram of the
image will also move slowly towards the right, resulting in a brighter
image.

    \hypertarget{problem-4-histogram-equalization-for-all-r-g-b-channels}{%
\section{Problem 4: Histogram equalization for all (R, G, B)
channels}\label{problem-4-histogram-equalization-for-all-r-g-b-channels}}

Very similar implementation as Problem 1.

    \begin{tcolorbox}[breakable, size=fbox, boxrule=1pt, pad at break*=1mm,colback=cellbackground, colframe=cellborder]
\prompt{In}{incolor}{122}{\boxspacing}
\begin{Verbatim}[commandchars=\\\{\}]
\PY{k}{def} \PY{n+nf}{eq\PYZus{}hist}\PY{p}{(}\PY{n}{arr}\PY{p}{)}\PY{p}{:}
    \PY{k}{return} \PY{n}{exposure}\PY{o}{.}\PY{n}{equalize\PYZus{}hist}\PY{p}{(}\PY{n}{arr}\PY{p}{)}

\PY{n}{src\PYZus{}red}\PY{p}{,} \PY{n}{src\PYZus{}green}\PY{p}{,} \PY{n}{src\PYZus{}blue} \PY{o}{=} \PY{n}{src\PYZus{}image}\PY{p}{[}\PY{p}{:}\PY{p}{,} \PY{p}{:}\PY{p}{,} \PY{l+m+mi}{0}\PY{p}{]}\PY{p}{,} \PY{n}{src\PYZus{}image}\PY{p}{[}\PY{p}{:}\PY{p}{,} \PY{p}{:}\PY{p}{,} \PY{l+m+mi}{1}\PY{p}{]}\PY{p}{,} \PY{n}{src\PYZus{}image}\PY{p}{[}\PY{p}{:}\PY{p}{,} \PY{p}{:}\PY{p}{,} \PY{l+m+mi}{2}\PY{p}{]}
\PY{n}{eq\PYZus{}red}\PY{p}{,} \PY{n}{eq\PYZus{}green}\PY{p}{,} \PY{n}{eq\PYZus{}blue} \PY{o}{=} \PY{n}{eq\PYZus{}hist}\PY{p}{(}\PY{n}{src\PYZus{}red}\PY{p}{)}\PY{p}{,} \PY{n}{eq\PYZus{}hist}\PY{p}{(}\PY{n}{src\PYZus{}green}\PY{p}{)}\PY{p}{,} \PY{n}{eq\PYZus{}hist}\PY{p}{(}\PY{n}{src\PYZus{}blue}\PY{p}{)}
\PY{c+c1}{\PYZsh{} Create the image}
\PY{n}{eq\PYZus{}image} \PY{o}{=} \PY{n}{np}\PY{o}{.}\PY{n}{dstack}\PY{p}{(}\PY{p}{(}\PY{n}{eq\PYZus{}red}\PY{p}{,} \PY{n}{eq\PYZus{}green}\PY{p}{,} \PY{n}{eq\PYZus{}blue}\PY{p}{)}\PY{p}{)}
\PY{c+c1}{\PYZsh{} Show the result}
\PY{n}{show\PYZus{}image}\PY{p}{(}\PY{n}{eq\PYZus{}image}\PY{p}{)}
\PY{n}{save\PYZus{}image}\PY{p}{(}\PY{l+s+s2}{\PYZdq{}}\PY{l+s+s2}{images/src\PYZus{}eq\PYZus{}hist.png}\PY{l+s+s2}{\PYZdq{}}\PY{p}{,} \PY{n}{eq\PYZus{}image}\PY{p}{)}
\end{Verbatim}
\end{tcolorbox}

    \begin{Verbatim}[commandchars=\\\{\}]
Lossy conversion from float64 to uint8. Range [0, 1]. Convert image to uint8
prior to saving to suppress this warning.
    \end{Verbatim}

    \begin{center}
    \adjustimage{max size={0.9\linewidth}{0.9\paperheight}}{main_files/main_24_1.png}
    \end{center}
    { \hspace*{\fill} \\}
    
    \begin{quote}
We can also apply \texttt{skimage.exposure.equalize\_hist} directly to
the 3-channel image, the result is NOT the same. This will be discussed
in \hyperref[problem-5-problems]{problem 5}.
\end{quote}

    \begin{tcolorbox}[breakable, size=fbox, boxrule=1pt, pad at break*=1mm,colback=cellbackground, colframe=cellborder]
\prompt{In}{incolor}{125}{\boxspacing}
\begin{Verbatim}[commandchars=\\\{\}]
\PY{c+c1}{\PYZsh{} Plot the histograms of all channels}
\PY{n}{plt}\PY{o}{.}\PY{n}{figure}\PY{p}{(}\PY{n}{dpi}\PY{o}{=}\PY{l+m+mi}{220}\PY{p}{,} \PY{n}{figsize}\PY{o}{=}\PY{p}{(}\PY{l+m+mi}{10}\PY{p}{,} \PY{l+m+mi}{4}\PY{p}{)}\PY{p}{)}
\PY{n}{plt}\PY{o}{.}\PY{n}{subplot}\PY{p}{(}\PY{l+m+mi}{1}\PY{p}{,} \PY{l+m+mi}{3}\PY{p}{,} \PY{l+m+mi}{1}\PY{p}{)}
\PY{n}{plt}\PY{o}{.}\PY{n}{hist}\PY{p}{(}\PY{n}{eq\PYZus{}red}\PY{o}{.}\PY{n}{flatten}\PY{p}{(}\PY{p}{)}\PY{p}{,} \PY{n}{bins}\PY{o}{=}\PY{l+m+mi}{256}\PY{p}{,} \PY{n+nb}{range}\PY{o}{=}\PY{p}{(}\PY{l+m+mi}{0}\PY{p}{,} \PY{l+m+mi}{1}\PY{p}{)}\PY{p}{,} \PY{n}{ec}\PY{o}{=}\PY{l+s+s1}{\PYZsq{}}\PY{l+s+s1}{red}\PY{l+s+s1}{\PYZsq{}}\PY{p}{,} \PY{n}{fc}\PY{o}{=}\PY{l+s+s1}{\PYZsq{}}\PY{l+s+s1}{red}\PY{l+s+s1}{\PYZsq{}}\PY{p}{)}
\PY{n}{plt}\PY{o}{.}\PY{n}{xlim}\PY{p}{(}\PY{l+m+mi}{0}\PY{p}{,} \PY{l+m+mi}{1}\PY{p}{)}
\PY{n}{plt}\PY{o}{.}\PY{n}{title}\PY{p}{(}\PY{l+s+s1}{\PYZsq{}}\PY{l+s+s1}{Equalized R}\PY{l+s+s1}{\PYZsq{}}\PY{p}{,} \PY{n}{loc}\PY{o}{=}\PY{l+s+s2}{\PYZdq{}}\PY{l+s+s2}{center}\PY{l+s+s2}{\PYZdq{}}\PY{p}{)}
\PY{n}{plt}\PY{o}{.}\PY{n}{subplot}\PY{p}{(}\PY{l+m+mi}{1}\PY{p}{,} \PY{l+m+mi}{3}\PY{p}{,} \PY{l+m+mi}{2}\PY{p}{)}
\PY{n}{plt}\PY{o}{.}\PY{n}{hist}\PY{p}{(}\PY{n}{eq\PYZus{}green}\PY{o}{.}\PY{n}{flatten}\PY{p}{(}\PY{p}{)}\PY{p}{,} \PY{n}{bins}\PY{o}{=}\PY{l+m+mi}{256}\PY{p}{,} \PY{n+nb}{range}\PY{o}{=}\PY{p}{(}\PY{l+m+mi}{0}\PY{p}{,} \PY{l+m+mi}{1}\PY{p}{)}\PY{p}{,} \PY{n}{ec}\PY{o}{=}\PY{l+s+s1}{\PYZsq{}}\PY{l+s+s1}{green}\PY{l+s+s1}{\PYZsq{}}\PY{p}{,} \PY{n}{fc}\PY{o}{=}\PY{l+s+s1}{\PYZsq{}}\PY{l+s+s1}{green}\PY{l+s+s1}{\PYZsq{}}\PY{p}{)}
\PY{n}{plt}\PY{o}{.}\PY{n}{xlim}\PY{p}{(}\PY{l+m+mi}{0}\PY{p}{,} \PY{l+m+mi}{1}\PY{p}{)}
\PY{n}{plt}\PY{o}{.}\PY{n}{title}\PY{p}{(}\PY{l+s+s1}{\PYZsq{}}\PY{l+s+s1}{Equalized G}\PY{l+s+s1}{\PYZsq{}}\PY{p}{,} \PY{n}{loc}\PY{o}{=}\PY{l+s+s2}{\PYZdq{}}\PY{l+s+s2}{center}\PY{l+s+s2}{\PYZdq{}}\PY{p}{)}
\PY{n}{plt}\PY{o}{.}\PY{n}{subplot}\PY{p}{(}\PY{l+m+mi}{1}\PY{p}{,} \PY{l+m+mi}{3}\PY{p}{,} \PY{l+m+mi}{3}\PY{p}{)}
\PY{n}{plt}\PY{o}{.}\PY{n}{hist}\PY{p}{(}\PY{n}{eq\PYZus{}blue}\PY{o}{.}\PY{n}{flatten}\PY{p}{(}\PY{p}{)}\PY{p}{,} \PY{n}{bins}\PY{o}{=}\PY{l+m+mi}{256}\PY{p}{,} \PY{n+nb}{range}\PY{o}{=}\PY{p}{(}\PY{l+m+mi}{0}\PY{p}{,} \PY{l+m+mi}{1}\PY{p}{)}\PY{p}{,} \PY{n}{ec}\PY{o}{=}\PY{l+s+s1}{\PYZsq{}}\PY{l+s+s1}{blue}\PY{l+s+s1}{\PYZsq{}}\PY{p}{,} \PY{n}{fc}\PY{o}{=}\PY{l+s+s1}{\PYZsq{}}\PY{l+s+s1}{blue}\PY{l+s+s1}{\PYZsq{}}\PY{p}{)}
\PY{n}{plt}\PY{o}{.}\PY{n}{xlim}\PY{p}{(}\PY{l+m+mi}{0}\PY{p}{,} \PY{l+m+mi}{1}\PY{p}{)}
\PY{n}{plt}\PY{o}{.}\PY{n}{title}\PY{p}{(}\PY{l+s+s1}{\PYZsq{}}\PY{l+s+s1}{Equalized B}\PY{l+s+s1}{\PYZsq{}}\PY{p}{,} \PY{n}{loc}\PY{o}{=}\PY{l+s+s2}{\PYZdq{}}\PY{l+s+s2}{center}\PY{l+s+s2}{\PYZdq{}}\PY{p}{)}
\PY{n}{plt}\PY{o}{.}\PY{n}{subplots\PYZus{}adjust}\PY{p}{(}\PY{n}{wspace}\PY{o}{=}\PY{l+m+mf}{0.2}\PY{p}{)}
\PY{n}{plt}\PY{o}{.}\PY{n}{show}\PY{p}{(}\PY{p}{)}
\end{Verbatim}
\end{tcolorbox}

    \begin{center}
    \adjustimage{max size={0.9\linewidth}{0.9\paperheight}}{main_files/main_26_0.png}
    \end{center}
    { \hspace*{\fill} \\}
    
    \hypertarget{problem-5-problems}{%
\section{Problem 5: Problems?}\label{problem-5-problems}}

\textbf{No!}

Sceptically speaking, problems only exist when we are facing a specific
\textbf{task}, but not a procedure. A procedure only has right or wrong,
but no \emph{good or bad}. Therefore, the problem itself is NOT
well-defined, which should not be present in the homework.

But let's experiment with the default RGB image histogram equalization
with our channel-wise histogram equalization. Are they identical?

    \begin{tcolorbox}[breakable, size=fbox, boxrule=1pt, pad at break*=1mm,colback=cellbackground, colframe=cellborder]
\prompt{In}{incolor}{130}{\boxspacing}
\begin{Verbatim}[commandchars=\\\{\}]
\PY{n}{eq\PYZus{}image\PYZus{}2} \PY{o}{=} \PY{n}{eq\PYZus{}hist}\PY{p}{(}\PY{n}{src\PYZus{}image}\PY{p}{)}
\PY{c+c1}{\PYZsh{} Compare the result with eq\PYZus{}image}
\PY{n}{delta\PYZus{}image} \PY{o}{=} \PY{n}{np}\PY{o}{.}\PY{n}{abs}\PY{p}{(}\PY{n}{eq\PYZus{}image} \PY{o}{\PYZhy{}} \PY{n}{eq\PYZus{}image\PYZus{}2}\PY{p}{)}
\PY{c+c1}{\PYZsh{} Show the result}
\PY{n}{show\PYZus{}image}\PY{p}{(}\PY{n}{delta\PYZus{}image}\PY{p}{)}
\end{Verbatim}
\end{tcolorbox}

    \begin{Verbatim}[commandchars=\\\{\}]
c:\textbackslash{}Program Files\textbackslash{}Python39\textbackslash{}lib\textbackslash{}site-packages\textbackslash{}skimage\textbackslash{}\_shared\textbackslash{}utils.py:394:
UserWarning: This might be a color image. The histogram will be computed on the
flattened image. You can instead apply this function to each color channel, or
set channel\_axis.
  return func(*args, **kwargs)
    \end{Verbatim}

    \begin{center}
    \adjustimage{max size={0.9\linewidth}{0.9\paperheight}}{main_files/main_28_1.png}
    \end{center}
    { \hspace*{\fill} \\}
    
    It can be seen that the channel-wise implementation is not identical to
the original implementation. What's happening under the hood?

    \hypertarget{reasoning}{%
\subsection{Reasoning}\label{reasoning}}

To understand how the difference occurs, we need to understand the
underlying process of histogram equalization.

\textbf{Histogram equalization is a non-linear process.} Channel
splitting and equalizing each channel separately is incorrect.
Equalization involves intensity values of the image, not the color
components. So for a simple RGB color image, histogram equalization
cannot be applied directly on the channels. It needs to be applied in
such a way that the intensity values are equalized \textbf{without
disturbing the color balance of the image}. So, the first step is to
convert the color space of the image from RGB into one of the color
spaces that separates intensity values from color components.

    \hypertarget{implementation}{%
\subsection{Implementation}\label{implementation}}

One good thing I choose Python instead of MATLAB is that I can see every
step of the process from the source code. It is obvious that
\texttt{skimage} feels something bad about computing histogram
equalization with a colored image, as it says:

\begin{verbatim}
UserWarning: This might be a color image. The histogram will be computed on the flattened image. You can instead apply this function to each color channel, or set channel_axis.
\end{verbatim}

But how? Diving into the source code, \texttt{skimage} implements
\texttt{equalize\_hist} as follows:

\begin{Shaded}
\begin{Highlighting}[]
\KeywordTok{def}\NormalTok{ equalize\_hist(image, nbins}\OperatorTok{=}\DecValTok{256}\NormalTok{, mask}\OperatorTok{=}\VariableTok{None}\NormalTok{):}
    \ControlFlowTok{if}\NormalTok{ mask }\KeywordTok{is} \KeywordTok{not} \VariableTok{None}\NormalTok{:}
\NormalTok{        mask }\OperatorTok{=}\NormalTok{ np.array(mask, dtype}\OperatorTok{=}\BuiltInTok{bool}\NormalTok{)}
\NormalTok{        cdf, bin\_centers }\OperatorTok{=}\NormalTok{ cumulative\_distribution(image[mask], nbins)}
    \ControlFlowTok{else}\NormalTok{:}
\NormalTok{        cdf, bin\_centers }\OperatorTok{=}\NormalTok{ cumulative\_distribution(image, nbins)}
\NormalTok{    out }\OperatorTok{=}\NormalTok{ np.interp(image.flat, bin\_centers, cdf)}
\NormalTok{    out }\OperatorTok{=}\NormalTok{ out.reshape(image.shape)}
    \ControlFlowTok{return}\NormalTok{ out.astype(utils.\_supported\_float\_type(image.dtype), copy}\OperatorTok{=}\VariableTok{False}\NormalTok{)}
\end{Highlighting}
\end{Shaded}

With no \texttt{mask} given, the function will calculate the cumulative
distribution function for the entire image. The function
\texttt{cumulative\_distribution} is defined as follows:

\begin{Shaded}
\begin{Highlighting}[]
\KeywordTok{def}\NormalTok{ cumulative\_distribution(image, nbins}\OperatorTok{=}\DecValTok{256}\NormalTok{):}
\NormalTok{    hist, bin\_centers }\OperatorTok{=}\NormalTok{ histogram(image, nbins)}
\NormalTok{    img\_cdf }\OperatorTok{=}\NormalTok{ hist.cumsum()}
\NormalTok{    img\_cdf }\OperatorTok{=}\NormalTok{ img\_cdf }\OperatorTok{/} \BuiltInTok{float}\NormalTok{(img\_cdf[}\OperatorTok{{-}}\DecValTok{1}\NormalTok{])}

\NormalTok{    cdf\_dtype }\OperatorTok{=}\NormalTok{ utils.\_supported\_float\_type(image.dtype)}
\NormalTok{    img\_cdf }\OperatorTok{=}\NormalTok{ img\_cdf.astype(cdf\_dtype, copy}\OperatorTok{=}\VariableTok{False}\NormalTok{)}

    \ControlFlowTok{return}\NormalTok{ img\_cdf, bin\_centers}
\end{Highlighting}
\end{Shaded}

And the function \texttt{histogram} is defined as follows:

\begin{Shaded}
\begin{Highlighting}[]
\AttributeTok{@utils.channel\_as\_last\_axis}\NormalTok{(multichannel\_output}\OperatorTok{=}\VariableTok{False}\NormalTok{)}
\KeywordTok{def}\NormalTok{ histogram(image, nbins}\OperatorTok{=}\DecValTok{256}\NormalTok{, source\_range}\OperatorTok{=}\StringTok{\textquotesingle{}image\textquotesingle{}}\NormalTok{, normalize}\OperatorTok{=}\VariableTok{False}\NormalTok{, }\OperatorTok{*}\NormalTok{,}
\NormalTok{              channel\_axis}\OperatorTok{=}\VariableTok{None}\NormalTok{):}
\NormalTok{    sh }\OperatorTok{=}\NormalTok{ image.shape}
    \ControlFlowTok{if} \BuiltInTok{len}\NormalTok{(sh) }\OperatorTok{==} \DecValTok{3} \KeywordTok{and}\NormalTok{ sh[}\OperatorTok{{-}}\DecValTok{1}\NormalTok{] }\OperatorTok{\textless{}} \DecValTok{4} \KeywordTok{and}\NormalTok{ channel\_axis }\KeywordTok{is} \VariableTok{None}\NormalTok{:}
\NormalTok{        utils.warn(}\StringTok{\textquotesingle{}This might be a color image. The histogram will be \textquotesingle{}}
                   \StringTok{\textquotesingle{}computed on the flattened image. You can instead \textquotesingle{}}
                   \StringTok{\textquotesingle{}apply this function to each color channel, or set \textquotesingle{}}
                   \StringTok{\textquotesingle{}channel\_axis.\textquotesingle{}}\NormalTok{)}

    \ControlFlowTok{if}\NormalTok{ channel\_axis }\KeywordTok{is} \KeywordTok{not} \VariableTok{None}\NormalTok{:}
\NormalTok{        channels }\OperatorTok{=}\NormalTok{ sh[}\OperatorTok{{-}}\DecValTok{1}\NormalTok{]}
\NormalTok{        hist }\OperatorTok{=}\NormalTok{ []}

        \CommentTok{\# compute bins based on the raveled array}
        \ControlFlowTok{if}\NormalTok{ np.issubdtype(image.dtype, np.integer):}
            \CommentTok{\# here bins corresponds to the bin centers}
\NormalTok{            bins }\OperatorTok{=}\NormalTok{ \_bincount\_histogram\_centers(image, source\_range)}
        \ControlFlowTok{else}\NormalTok{:}
            \CommentTok{\# determine the bin edges for np.histogram}
\NormalTok{            hist\_range }\OperatorTok{=}\NormalTok{ \_get\_numpy\_hist\_range(image, source\_range)}
\NormalTok{            bins }\OperatorTok{=}\NormalTok{ \_get\_bin\_edges(image, nbins, hist\_range)}

        \ControlFlowTok{for}\NormalTok{ chan }\KeywordTok{in} \BuiltInTok{range}\NormalTok{(channels):}
\NormalTok{            h, bc }\OperatorTok{=}\NormalTok{ \_histogram(image[..., chan], bins, source\_range, normalize)}
\NormalTok{            hist.append(h)}
        \CommentTok{\# Convert to numpy arrays}
\NormalTok{        bin\_centers }\OperatorTok{=}\NormalTok{ np.asarray(bc)}
\NormalTok{        hist }\OperatorTok{=}\NormalTok{ np.stack(hist, axis}\OperatorTok{=}\DecValTok{0}\NormalTok{)}
    \ControlFlowTok{else}\NormalTok{:}
\NormalTok{        hist, bin\_centers }\OperatorTok{=}\NormalTok{ \_histogram(image, nbins, source\_range, normalize)}

    \ControlFlowTok{return}\NormalTok{ hist, bin\_centers}
\end{Highlighting}
\end{Shaded}

If \texttt{channel\_axis} is not set, the histogram is computed on the
flattened image. For color or multichannel images, set
\texttt{channel\_axis} to use a common binning for all channels.
Alternatively, one may apply the function separately on each channel to
obtain a histogram for each color channel with separate binning.

A \emph{flattened} image is a combined array of all the pixels of each
channel in the image. This means that all 3 channels are taken into
consideration when computing the histogram. This helps to avoid the
problem of color balance, making one specific channel too bright than
the other two.

So it seems that the RGB image should not be histogram-equalized by
simply separating the channels. What should we do instead?

    \hypertarget{solution}{%
\subsection{Solution}\label{solution}}

The answer is an alternative color space. There are many different color
spaces aside from RGB, each with its own set of color components. For
example, the CIE XYZ color space has 3 components, while the CIE L*a*b*
color space has 4.

Some of the possible options are HSV/HLS, YUV, YCbCr, etc.. YCbCr is
preferred as it is designed for digital images. Perform histogram
equalization on the intensity plane Y, and then convert the resultant
YCbCr image back to RGB.

    \begin{tcolorbox}[breakable, size=fbox, boxrule=1pt, pad at break*=1mm,colback=cellbackground, colframe=cellborder]
\prompt{In}{incolor}{137}{\boxspacing}
\begin{Verbatim}[commandchars=\\\{\}]
\PY{k}{def} \PY{n+nf}{RGB2YCbCr}\PY{p}{(}\PY{n}{ch\PYZus{}r}\PY{p}{,} \PY{n}{ch\PYZus{}g}\PY{p}{,} \PY{n}{ch\PYZus{}b}\PY{p}{)}\PY{p}{:}
    \PY{c+c1}{\PYZsh{} Based on ITU\PYZhy{}R BT.601 standard}
    \PY{n}{Y}  \PY{o}{=} \PY{n}{ch\PYZus{}r} \PY{o}{*}  \PY{l+m+mf}{0.29900} \PY{o}{+} \PY{n}{ch\PYZus{}g} \PY{o}{*}  \PY{l+m+mf}{0.58700} \PY{o}{+} \PY{n}{ch\PYZus{}b} \PY{o}{*}  \PY{l+m+mf}{0.11400}
    \PY{n}{Cb} \PY{o}{=} \PY{n}{ch\PYZus{}r} \PY{o}{*} \PY{o}{\PYZhy{}}\PY{l+m+mf}{0.16874} \PY{o}{+} \PY{n}{ch\PYZus{}g} \PY{o}{*} \PY{o}{\PYZhy{}}\PY{l+m+mf}{0.33126} \PY{o}{+} \PY{n}{ch\PYZus{}b} \PY{o}{*}  \PY{l+m+mf}{0.50000} \PY{o}{+} \PY{l+m+mi}{128}
    \PY{n}{Cr} \PY{o}{=} \PY{n}{ch\PYZus{}r} \PY{o}{*}  \PY{l+m+mf}{0.50000} \PY{o}{+} \PY{n}{ch\PYZus{}g} \PY{o}{*} \PY{o}{\PYZhy{}}\PY{l+m+mf}{0.41869} \PY{o}{+} \PY{n}{ch\PYZus{}b} \PY{o}{*} \PY{o}{\PYZhy{}}\PY{l+m+mf}{0.08131} \PY{o}{+} \PY{l+m+mi}{128}
    \PY{k}{return} \PY{n}{Y}\PY{p}{,} \PY{n}{Cb}\PY{p}{,} \PY{n}{Cr}

\PY{k}{def} \PY{n+nf}{YCbCr2RGB}\PY{p}{(}\PY{n}{ch\PYZus{}y}\PY{p}{,} \PY{n}{ch\PYZus{}cb}\PY{p}{,} \PY{n}{ch\PYZus{}cr}\PY{p}{)}\PY{p}{:}
    \PY{c+c1}{\PYZsh{} Based on ITU\PYZhy{}R BT.601 standard}
    \PY{n}{R}  \PY{o}{=} \PY{n}{ch\PYZus{}y} \PY{o}{+}                          \PY{o}{+} \PY{p}{(}\PY{n}{ch\PYZus{}cr} \PY{o}{\PYZhy{}} \PY{l+m+mi}{128}\PY{p}{)} \PY{o}{*}  \PY{l+m+mf}{1.40200}
    \PY{n}{G}  \PY{o}{=} \PY{n}{ch\PYZus{}y} \PY{o}{+} \PY{p}{(}\PY{n}{ch\PYZus{}cb} \PY{o}{\PYZhy{}} \PY{l+m+mi}{128}\PY{p}{)} \PY{o}{*} \PY{o}{\PYZhy{}}\PY{l+m+mf}{0.34414} \PY{o}{+} \PY{p}{(}\PY{n}{ch\PYZus{}cr} \PY{o}{\PYZhy{}} \PY{l+m+mi}{128}\PY{p}{)} \PY{o}{*} \PY{o}{\PYZhy{}}\PY{l+m+mf}{0.71414}
    \PY{n}{B}  \PY{o}{=} \PY{n}{ch\PYZus{}y} \PY{o}{+} \PY{p}{(}\PY{n}{ch\PYZus{}cb} \PY{o}{\PYZhy{}} \PY{l+m+mi}{128}\PY{p}{)} \PY{o}{*}  \PY{l+m+mf}{1.77200}
    \PY{k}{return} \PY{n}{R}\PY{p}{,} \PY{n}{G}\PY{p}{,} \PY{n}{B}
\end{Verbatim}
\end{tcolorbox}

    \begin{tcolorbox}[breakable, size=fbox, boxrule=1pt, pad at break*=1mm,colback=cellbackground, colframe=cellborder]
\prompt{In}{incolor}{159}{\boxspacing}
\begin{Verbatim}[commandchars=\\\{\}]
\PY{c+c1}{\PYZsh{} Convert src\PYZus{}img to YCbCr}
\PY{n}{src\PYZus{}ycrcb} \PY{o}{=} \PY{n}{cv2}\PY{o}{.}\PY{n}{cvtColor}\PY{p}{(}\PY{n}{src\PYZus{}image}\PY{p}{,} \PY{n}{cv2}\PY{o}{.}\PY{n}{COLOR\PYZus{}RGB2YCrCb}\PY{p}{)}
\PY{n}{src\PYZus{}y}\PY{p}{,} \PY{n}{src\PYZus{}cr}\PY{p}{,} \PY{n}{src\PYZus{}cb} \PY{o}{=} \PY{n}{src\PYZus{}ycrcb}\PY{p}{[}\PY{p}{:}\PY{p}{,} \PY{p}{:}\PY{p}{,} \PY{l+m+mi}{0}\PY{p}{]}\PY{p}{,} \PY{n}{src\PYZus{}ycrcb}\PY{p}{[}\PY{p}{:}\PY{p}{,} \PY{p}{:}\PY{p}{,} \PY{l+m+mi}{1}\PY{p}{]}\PY{p}{,} \PY{n}{src\PYZus{}ycrcb}\PY{p}{[}\PY{p}{:}\PY{p}{,} \PY{p}{:}\PY{p}{,} \PY{l+m+mi}{2}\PY{p}{]}
\PY{c+c1}{\PYZsh{} Equalize histogram on Y channel}
\PY{n}{eq\PYZus{}y} \PY{o}{=} \PY{n}{eq\PYZus{}hist}\PY{p}{(}\PY{n}{src\PYZus{}y}\PY{p}{)} \PY{o}{*} \PY{l+m+mi}{255}
\PY{c+c1}{\PYZsh{} Convert back to RGB}
\PY{n}{eq\PYZus{}ycbcr} \PY{o}{=} \PY{n}{np}\PY{o}{.}\PY{n}{dstack}\PY{p}{(}\PY{p}{(}\PY{n}{eq\PYZus{}y}\PY{p}{,} \PY{n}{src\PYZus{}cr}\PY{p}{,} \PY{n}{src\PYZus{}cb}\PY{p}{)}\PY{p}{)}
\PY{n}{eq\PYZus{}ycbcr} \PY{o}{=} \PY{n}{np}\PY{o}{.}\PY{n}{uint8}\PY{p}{(}\PY{n}{eq\PYZus{}ycbcr}\PY{p}{)}
\PY{n}{eq\PYZus{}image\PYZus{}from\PYZus{}ycbcr} \PY{o}{=} \PY{n}{cv2}\PY{o}{.}\PY{n}{cvtColor}\PY{p}{(}\PY{n}{eq\PYZus{}ycbcr}\PY{p}{,} \PY{n}{cv2}\PY{o}{.}\PY{n}{COLOR\PYZus{}YCrCb2RGB}\PY{p}{)}
\PY{c+c1}{\PYZsh{} Show the result}
\PY{n}{show\PYZus{}image}\PY{p}{(}\PY{n}{eq\PYZus{}image\PYZus{}from\PYZus{}ycbcr}\PY{p}{)}
\end{Verbatim}
\end{tcolorbox}

    \begin{center}
    \adjustimage{max size={0.9\linewidth}{0.9\paperheight}}{main_files/main_34_0.png}
    \end{center}
    { \hspace*{\fill} \\}
    
    We see that the problem of a color tinting has been resolved. As the
Y-channel is the intensity channel, the overall color balance is
maintained, which gives a better result than the original per-channel
histogram equalization.

    \hypertarget{conclusion}{%
\section{Conclusion}\label{conclusion}}

That's all for this assignment. During the whole task, we've toured
through image processing and how to match image histograms with a
specific distribution. We also tried to compare channel-wise histogram
equalization with the Y-channel histogram equalization, based on the
fact that the Y-channel is the intensity channel. We see a significant
image quality boost as we switched our color space from RGB to YCbCr.

The world of DSP is growing rapidly. There are many tools and algorithms
to process and enhance imges, and we shall continue to explore them.

It's quite a pity that this assignment may be the last programming
assignment. If you are a student who is also interested in DSIP,
congratulations for covering this repo, and thanks for your time. I hope
this repository will be useful to you.

\hypertarget{license}{%
\section{License}\label{license}}

The above code is under MIT License.


    % Add a bibliography block to the postdoc
    
    
    
\end{document}
